\documentclass[12pt, openany, letterpaper]{memoir}
\usepackage{NotesStyle}

\begin{document}
\title{CHEM 1110 Lecture Notes}
\author{Matthew Rowley}
\mainmatter
\maketitle

\chapter*{Course Administrative Details}
\begin{itemize}
	\item My office hours
	\item Intro to my research
	\item Introductory Quiz
	\item Grading details
	      \begin{itemize}
		      \item Exams - 40, Final - 20, Quizzes - 10, Achieve Homework - 15, Textbook Homework - 15
		      \item Online Homework
		      \item Textbook Homework
		      \item Daily quizzes
	      \end{itemize}
	\item Importance of reading and learning on your own
	\item Learning resources
	      \begin{itemize}
		      \item My Office Hours
		      \item Tutoring services - https://www.suu.edu/academicsuccess/tutoring/
	      \end{itemize}
	\item Show how to access Canvas
	      \begin{itemize}
		      \item Calendar, Grades, Modules, etc.
		      \item Achieve Homework
	      \end{itemize}
\end{itemize}

\chapter{Matter and Measurements}
\section{Chemistry: The Central Science}
\begin{itemize}
	\item My demos (Ruby, Periodic Table, NaAc hand warmers, Electroplating)
	\item Chemistry is the study of \emph{matter}, its \emph{properties}, and modes of \emph{changes}
	\item Chemistry is the ``Central Science'' because it ties in to other disciplines
	\item Figure 1.1 -- Web of disciplines
	\item Physical and chemical properties
	\item Physical and chemical changes
	      \begin{itemize}
		      \item Pop balloon vs explode balloon
	      \end{itemize}
\end{itemize}

\section{States of Matter}
\begin{itemize}
	\item Three phases of matter
	      \begin{description}
		      \item[Solid:] Definite shape and volume -- Particles are held rigidly in place
		      \item[Liquid:] Changing shape but definite volume -- Particles flow past one another
		      \item[Gas:] Variable shape and volume -- Particles fly apart and fill all available space
	      \end{description}
	\item State changes: Fusion/Freezing, Vaporization/Condensation, Sublimation/Deposition
\end{itemize}

\section{Classification of Matter}
\begin{itemize}
	\item Figure 1.3 -- flow chart
	\item Pure substances vs mixtures
	\item Homogeneous vs heterogeneous mixtures
	\item Compounds vs Elements
	\item Physical and chemical changes on the flow chart
	\item How do these changes look at the atomic level (un-numbered figure on p.6)
	\item Anatomy of a chemical reaction
\end{itemize}

\section{Chemical Elements and Symbols}
\begin{itemize}
	\item Elements are given names and symbols (Some, like Fe, don't match in English)
	\item Table 1.2 -- Names and symbols of common elements
	\item Table 1.3 and 1.4 -- Elements in the Earth and in our bodies
	\item Chemical formulas -- Type and amount of each element in a compound (\ch{H2O}, \ch{C12H22O11}, etc.)
\end{itemize}

\section{Chemical Reactions: An Example of Chemical Change}
\begin{itemize}
	\item Reactants, and Products
	\item Reaction arrow and writing chemical equations
	\item Demo: Zinc in copper nitrate solution
\end{itemize}

\section*{Homework 1.1:}
\begin{itemize}
	\item 1.36 -- Physical vs. Chemical changes
	\item 1.38 -- Changes of state
	\item 1.41 -- Mixtures vs. Pure substances
	\item 1.44 -- Anatomy of chemical reaction
	\item 1.46 -- Element symbols
	\item 1.50 -- Counting atoms in a formula
	\item 1.52 -- Writing a formula from composition
\end{itemize}

\section{Physical Quantities: Units and Scientific Notation}
\begin{itemize}
	\item Physical quantities are things that can be measured or calculated
	\item SI units (Table 1.5)
	\item Metric prefixes and $10^n$ conversions (Table 1.6)
	\item Derived units
	      \begin{itemize}
		      \item Squared and cubed units ($cm^3$)
		      \item Combined units ($\nicefrac{m}{s}$)
	      \end{itemize}
	      \begin{itemize}
		      \item Each prefix is a multiplication operation
		      \item Why would you use prefixes?
		      \item $100m$, $0.1km$, and $10,000cm$
	      \end{itemize}
	\item Using units to check your formulas: $B_e = \dfrac{h}{8\pi^2\mu R_e^2c}=cm^{-1}$
	\item Scientific Notation
	      \begin{itemize}
		      \item Expressing numbers in scientific notation
		            \begin{itemize}
			            \item Split the number into \emph{quantity} and \emph{magnitude}
			            \item Express the quantity as a number with MSD in the ones place
			            \item Express the magnitude as a power of $10$
		            \end{itemize}
		      \item Scientific notation and metric prefixes
		            \begin{itemize}
			            \item For metric to scientific: ($23.6kg\rightarrow2.36\times10^4g$)

			                  First convert the metric prefix to a power of 10, then shift the decimal of the quantity to put its MSD in the ones place
			            \item For scientific to metric: ($5.60\times10^{-7}m\rightarrow560nm$)

			                  First shift the decimal to make the power of 10 match a prefix, then replace that power of 10 with the prefix
		            \end{itemize}
	      \end{itemize}
\end{itemize}

\section{Measuring Mass, Length, and Volume}
\begin{itemize}
	\item Weight vs mass
	\item kg,l (SI units) vs g,ml (More convenient in the lab)
	\item SI vs other systems (Tables 1.7, 1.8, and 1.9)
	\item Figure 1.6 shows the large impact of cubed units (i.e. $1m^3 = 1000000cm^3$)
\end{itemize}

\section{Measurement and Significant Figures}
\begin{itemize}
	\item Accuracy in measurements
	\item Statistics ($\sigma$) in many measurements, simple rules for one measurement
	\item Consider my height: $6ft$ vs $74in$ vs $73.784672in$
	      \begin{itemize}
		      \item More digits implies a more precise measurement
		      \item The first one is correct, if we allow that it is just approximate
		      \item The last one is only appropriate if I was \emph{actually} measured to the millionth of an inch
		      \item If I were precisely $6ft$ (with high precision), how would I write that? ($6.000ft$)
	      \end{itemize}
	\item Figures 1.6-7 show how we report measurements with balances and graduated glassware
	\item Significant figures
	      \begin{itemize}
		      \item Zeros in the middle of two numbers are significant
		      \item Zeros at the beginning of a number are never significant
		      \item Zeros at the end of a number \emph{after} the decimal point are significant
		      \item Zeros at the end of a number \emph{before} the decimal point are not significant
	      \end{itemize}
	\item Scientific notation makes it easy to express precision (every $0$ in scientific notation is significant)
	\item Consider $2,300kg$ measured to $\pm1kg$ ($2.300\times10^3g$)
	\item We should keep track of the total number of significant figures, and the position of the \emph{least} significant figure in a number
	\item Sig-fig practice: Give \# of SFs and LSD position for several numbers
\end{itemize}


\section{Rounding off Numbers}
\begin{itemize}
	\item In science, we do not round off numbers haphazardly; Rounding reflects precision
	\item Error analysis for addition and subtraction
	      \begin{itemize}
		      \item The LSD of the answer will match the LSD of the least precise input
		      \item \# of SFs doesn't matter, and the answer might gain or lose SFs compared to the inputs
		      \item Practice: $96g+43g=139g$ \hspace{1em}and\hspace{1em} $23.4g-18.6g=4.8g$
		      \item Consider adding $1.00\mu l$ of water to a $8L$ Ace Hardware bucket -- The precision of the final amount is entirely determined by the low precision of the bucket
	      \end{itemize}
	\item Error analysis for multiplication and division
	      \begin{itemize}
		      \item The \# of SFs in the answer will match the smallest \# of SFs from the inputs
		      \item Here, place value does not matter at all
		      \item Practice: $45.7g \div 8.2cm^3 = 5.6\nicefrac{g}{cm^3}$ \hspace{1em}and\hspace{1em} $82.5\nicefrac{miles}{hour}\times53.24hours=4,390miles$
	      \end{itemize}
	\item Compound problems
	      \begin{itemize}
		      \item Solve compound problems step-by-step, writing the intermediate answers
		      \item For each intermediate answer, keep track of the \# of SFs and LSD position
		      \item Only round the final answer, don't round intermediates
		      \item Practice: $\frac{12.3g+34g}{12.0cm^3+7.7cm^3}=2.4\nicefrac{g}{cm^3}$ (wrong answer with premature rounding)
	      \end{itemize}
\end{itemize}

\section*{Homework 1.2:}
\begin{itemize}
	\item 1.58 -- Units and metric prefixes
	\item 1.60 -- Scientific Notation
	\item 1.62 -- Counting Significant Figures
	\item 1.66 -- +- error propogation
	\item 1.67 -- */ error propogation
\end{itemize}

\section{Problem Solving: Unit Conversions and Estimating Answers}
\begin{itemize}
	\item Conversion factors
	      \begin{itemize}
		      \item Rearrange conversion equation to make unity ratios: $1=\frac{1in}{2.54cm}=\frac{2.54cm}{1in}$
		      \item Multiply a number by the right ratio to cancel out the starting units and leave the converted units
		      \item ``Do I multiply or divide by the conversion factor?'' is the wrong question. You are always multiplying by a ratio. The right question is ``Where should the units be in my conversion factor?'' The numbers go along with the units
		      \item Most conversion factors have ``$1$'' with infinite significant figures, and the other number limits SFs in a calculation
		      \item Some conversions are defined as \emph{perfect} (such as $60s=1min$) and won't limit SFs at all
		      \item Take care to square or cube the first-power ratio for square or cubed dimensions:

		            $1in=2.54cm\rightarrow1^3in^3=\left(2.54\right)^3cm^3\rightarrow 1in^3=16.4cm^3$
	      \end{itemize}
	\item Dimensional analysis is a method which frames any problem (even complex ones) in terms of multiple unit conversions
	      \begin{itemize}
		      \item The railroad ties or picket fence method makes it easy to organize a problem
		      \item Set up all the unit conversions before doing any calculations
		      \item Sort out significant figures at the end
		      \item Practice: $\frac{65.3miles}{h}\left|\frac{1h}{60min}\right|\frac{1min}{60s}\left|\frac{1.61km}{1mile}\right|\frac{1000m}{1km}=28.4\nicefrac{m}{s}$
	      \end{itemize}
\end{itemize}

\section{Temperature, Heat, and Energy}
\begin{itemize}
	\item All chemical reactions involve a change in energy
	\item Energy is the capacity to supply heat or do work. It is in units of J
	\item Temperature is the measure of heat energy in an object
	      \begin{itemize}
		      \item We don't use F (Booooooo, Fahrenheit!)
		      \item $^\circ$C is split so that $0^\circ$C is freezing and $100^\circ$C is boiling
		      \item Absolute temperature, K, has spacing equal to $^\circ$C, but starts at absolute zero
		      \item Absolute zero is the temperature when all heat has been removed. The $0$ in Kelvin is more fundamental than the $0$ in $^\circ$C
		      \item $T( ^\circ C) = T(K)-273.15$
	      \end{itemize}
	\item The energy needed to raise the temperature of a substance is its specific heat (Table 1.10)
	\item $1cal=4.184J$, based on the heat capacity of water: $4.184\nicefrac{J}{gK}$
	\item $q=mC\Delta T$
	      \begin{itemize}
		      \item How many J of heat are required to raise the temperature of $2.5~g$ of gold by $8.0^\circ C$? ($2.6~J$)
		      \item $23.5~J$ of heat are removed from $44~g$ of iron. What is the temperature change? ($-1.2^\circ C$)
	      \end{itemize}
\end{itemize}

\section{Density and Specific Gravity}
\begin{itemize}
	\item $density=\dfrac{mass}{volume}$ (Table 1.11), and can be used to convert mass and volume
	      \begin{itemize}
		      \item A classroom has a volume of $8.00\times 10^3ft^3$. What is the mass of the air contained in that room? ($268,000g$)
		      \item A sample of gold weighs $8.5~g$. What is the volume of this sample? ($0.44~cm^3$)
	      \end{itemize}
	\item Density can be affected by temperature changes and phase changes. Water exhibits anomalous behavior in this respect
	\item Specific gravity relates the density of a substance to the density of water, and is measured by a hydrometer (Figure 1.10)
\end{itemize}

\section*{Homework 1.3:}
\begin{itemize}
	\item 1.78 -- Dimensional analysis word problem
	\item 1.82 -- Specific heat
	\item 1.88 -- Density measurement
\end{itemize}

\chapter{Atoms and the Periodic Table}%Chapter 2
\section{Atomic Theory}
\begin{itemize}
	\item DEMO -- How many times can we subdivide a crystal of salt and have it still be salt?
	\item Postulates of atomic theory:
	      \begin{itemize}
		      \item All matter is composed of atoms
		      \item The atoms of a given element differ from the atoms of all other elements
		      \item Chemical compounds consist of atoms combined in specific ratios (\ch{H2O} vs \ch{H2O2})
		      \item Chemical reactions change the way that atoms are combined in compounds, but leave the atoms themselves unchanged
	      \end{itemize}
	\item Today we know that atoms are very small ($m=10^{-23}g$ and $r=10^{-10}m$)
	\item Atomic mass unit
	      \begin{itemize}
		      \item AMU is approximately the mass of a proton or a neutron
		      \item Electrons are very small (less than a thousandth of an AMU)
		      \item Technically, \ch{^{12}C} has a mass of \emph{exactly} $12amu$
	      \end{itemize}
	\item Subatomic particles and their interactions (Table 2.1)
	\item Atomic structure (Figure 2.1)
	      \begin{itemize}
		      \item The protons and neutrons make up an extremely dense nucleus
		      \item Electrons occupy the rest (more on this later)
		      \item Most of the volume of atoms is empty (well, filled only with very low mass electrons)
	      \end{itemize}
\end{itemize}
\section{Elements and Atomic Number}
\begin{itemize}
	\item Atomic number (Z) is the number of protons and it distinguishes one element from another
	\item Neutral atoms will have the same number of electrons as protons
	\item Charged atoms (either more or fewer electrons than protons) are called ions
	\item Mass number (A) is the number of protons \emph{and} neutrons
\end{itemize}
\section{Isotopes and Atomic Weight}
\begin{itemize}
	\item Atoms with the same atomic number, but different mass numbers are called isotopes
	\item Hydrogen, deuterium, and tritium are the three named isotopes of hydrogen
	\item Atomic symbols: \ch{^{3}_{1}H}
	      \begin{itemize}
		      \item The mass number is in the top-left
		      \item The atomic number (can be omitted) is in the bottom left
		      \item The charge is in the top right (omitted if neutral)
	      \end{itemize}
	\item Atomic weight -- The non-integer number on the periodic table
	      \begin{itemize}
		      \item Atomic weight is based on the precise mass and natural abundances of the isotopes
		      \item $AW=\sum\limits_{isotopes}\left(mass\times \frac{\%~abundance}{100\%}\right)$
	      \end{itemize}
	\item \ch{^6Li}: $7.59\%$ abundance and $m=6.015122amu$. \ch{^7Li}: $92.41\%$ abundance and $m=7.016004amu$
\end{itemize}

\section*{Homework 2.1:}
\begin{itemize}
	\item 2.32 -- How elements differ from each other
	\item 2.42 -- Composition from atomic symbol
	\item 2.46 -- Atomic symbols from composition
	\item 2.48 -- Calculate atomic mass from isotopes
\end{itemize}

\section{The Periodic Table}
\begin{itemize}
	\item As more elements were discovered, scientists tried to find commonalities and patterns
	\item \ch{Li}, \ch{Na}, and \ch{K} were similar to each other, as were \ch{Cl}, \ch{Br}, and \ch{I}
	\item Mendeleev arranged the elements in increasing mass, and saw a pattern
	\item Mendeleev even correctly predicted the properties of Gallium, Germanium, and Scandium -- yet-undiscovered elements
	\item Rows are called ``Periods'' and columns are called ``Groups''
	\item The periods get longer further down due to electronic structure -- We will talk about this soon
	\item Figure 2.2 shows two ways to divide the periodic table
	      \begin{itemize}
		      \item Main group vs. transition metals vs. inner transition metals
		      \item Metals vs. Metalloids vs. Non-metals
	      \end{itemize}
\end{itemize}

\section{Some Characteristics of Different Groups}
\begin{itemize}
	\item Periodic trend in atomic radius (Figure 2.3)
	\item Group 1A -- Alkali metals: Shiny, soft, and highly reactive
	\item Group 2A -- Alkaline earth metals: Like Alkali, but less extreme
	\item Group 7A -- Halogens: Colorful and corrosive in elemental form
	\item Group 8A -- Noble gases: Won't react with anything (*almost*)
\end{itemize}

\section{Electronic Structure of Atoms}
\begin{itemize}
	\item Electronic structure largely determines physical and chemical properties
	\item Electrons can only exist in quantized energy states because of their quantum wavelike nature
	\item These states have different \emph{sizes} and characteristic \emph{shapes} in 3-D space. The state defines where an electron can be found, and how much energy it has
	\item Each state of an electron can be characterized by \emph{quantum numbers}, which can be thought of as the ``address'' for a given electron
	      \begin{itemize}
		      \item First is the shell number, $n$
		            \begin{itemize}
			            \item The shell determines the energy and size of a state
			            \item Shells can hold one or more subshells
			            \item Each successive shell can hold more electrons than the last (2, 8, 18, 32)
		            \end{itemize}
		      \item  Next is the subshell number, $l$
		            \begin{itemize}
			            \item Subshells come in 4 types: $s$, $p$, $d$, and $f$
			            \item The subshell type determines the shape of the electron state
			            \item Each subshell has a different number of individual states (orbitals), and each state can contain 2 electrons

			                  \begin{tabular}{c|c|c|c}
				                  Subshell & Shape    & \# of Orbitals & Max \# of electrons \\ \midrule
				                  $s$      & Sphere   & $1$            & $2$                 \\
				                  $p$      & Dumbell  & $3$            & $6$                 \\
				                  $d$      & Clover   & $5$            & $10$                \\
				                  $f$      & Complex! & $7$            & $14$
			                  \end{tabular}
		            \end{itemize}
		      \item Each successive shell adds a new subshell type (Table 2.2)
		      \item Spin differentiates the two electrons which share an orbital
	      \end{itemize}
	\item Draw a hydrogenic energy level diagram, and practice identifying shells and subshells
\end{itemize}

\section{Electron Configurations}
\begin{itemize}
	\item For multi-electron atoms, the subshells arrange themselves a bit differently
	      \begin{itemize}
		      \item The first few subshells are: $1s$, $2s$, $2p$, $3s$, $3p$, $4s$, $3d$, $4p$ (Draw the energy level diagram)
		      \item We will show a way to remember this pattern in just a few minutes
	      \end{itemize}
	\item Electrons occupy orbitals according to the following rules:
	      \begin{itemize}
		      \item Electrons will occupy the lowest energy orbital available. This is the Aufbau principle.
		      \item Each orbital can hold only two electrons, which must have opposite spins. This is the Pauli exclusion principle
		      \item Degenerate groups of orbitals are all filled halfway with electrons of the same spin before any electrons are paired up. This is Hund's rule
	      \end{itemize}
	\item To find the configuration of an element, count the number of electrons and fill the orbitals according to the three rules above
	\item To give the configuration, list the occupied subshells in order, and their occupancy as a superscript
	\item Practice: \ch{O:}$1s^22s^22p^4$ \hspace{2em} \ch{Si:}$1s^22s^2sp^63s^23p^2$
	\item Table 2.3 shows electron configurations for the first 20 elements
	\item Noble gas notation
	      \begin{itemize}
		      \item For longer configurations, we can reference the noble gas which comes previous to the element
		      \item The electrons represented by that noble gas are \emph{core} electrons, and are not actually very important to bonding, ion formation etc.
		      \item The remaining electrons are in the outermost, \emph{valence} shell, and are most important to chemistry
		      \item Practice: \ch{Cl: [Ne]}$3s^23p^5$
	      \end{itemize}
\end{itemize}

\section{Electron Configurations and the Periodic Table}
\begin{itemize}
	\item The periodic table is arranged according to properties, but properties are governed by electronic configuration. Ergo, the periodic table is arranged according to electronic configuration!
	\item Figure 2.7 shows how the blocks of the periodic table represent subshells in electronic structure
	\item My figure includes $d(n-1)$ and $f(n-2)$ labels
	\item Note that the width of each block matches the number of electrons each subshell type can hold
	\item The order of the subshells can be found by simply traversing up the elements in the periodic table
	\item Practice: \ch{Pb:[Xe]}$6s^24f^{14}5d^{10}6p^2$
\end{itemize}

\section{Electron Dot Symbols}
\begin{itemize}
	\item It is useful to graphically represent the valence electrons with electron dot symbols
	\item Write the symbol for the element, then surround it with the correct number of dots to represent valence electrons
	\item Dots go on the four sides of the symbol, and pair up only when there are more than 4 electrons
	\item Table 2.5 shows many of these diagrams
\end{itemize}

\section*{Homework 2.2:}
\begin{itemize}
	\item 2.56 -- Classify elements (metal, non-metal, main group, etc.)
	      % \item 2.62 -- Reactive trends in the periodic table
	\item 2.66 -- How many subshells in each shell
	\item 2.70 -- Electron arrow diagrams
	\item 2.73 -- Electronic configurations of elements
\end{itemize}

\chapter{Ionic Compounds}
\section{Ions}
\begin{itemize}
	\item Metals and non-metals will combine to form ionic compounds
	\item Ionic compounds have high melting points, stable crystal structures, and a degree of water solubility
	\item When ionic compounds dissolve in water, they conduct electricity
	\item Positive ions are called \emph{cations} and negative ions are called \emph{anions}
	\item Many reactions involve ions which are already charged, while other reactions generate ions through gaining or losing electrons
\end{itemize}

\section{Ions and the Octet Rule}
\begin{itemize}
	\item Elements will tend to gain or lose enough electrons to have a full \emph{octet} of valence electrons
	\item This means that non-metals will gain electrons to reach the electron configuration of the next highest noble gas
	\item Metals will lose electron to reach the next lowest noble gas
	\item Electron configurations for ions are written just the same as electron configurations for elements
\end{itemize}

\section{Ions of Some Common Elements}
\begin{itemize}
	\item Use the periodic table to predict the charge an element will take on as it forms an ion
	\item Group 1A will be 1+
	\item Group 2A will be 2+
	\item Group 6A will be 2-
	\item Group 7A will be 1-
\end{itemize}

\section{Periodic Properties and Ion Formation}
\begin{itemize}
	\item Ionization energy is the energy required to remove an electron and create a cation
	\item Electron affinity is the energy given off when an electron is gained, making an anion
	\item Periodic trends in ionization energy and electron affinity
	\item Nonmetals such as \ch{C} and \ch{N} tend to not form ions
\end{itemize}

\section{Naming Ions}
\begin{itemize}
	\item For metals with just one ion (1A, 2A, and 3A groups), just add ``ion'' to the end of the name
	\item For metals with multiple ions (transition metals, 4A, and 5A), state the charge and then add ``ion''
	\item An old rule uses different endings (ferrous and chromous for $2+$, and ferric and chromic for $3+$)
	\item Anions change the end of the element name to ``-ide''
	\item See tables 3.1 and 3.2
\end{itemize}

\section*{Homework 3.1:}
\begin{itemize}
  \item 3.42 -- Identify ion from charge and electron number
  \item 3.44 -- Ions and the octet rule
  \item 3.46 -- Ion formation by gaining/losing electrons
  \item 3.50 -- Ionization energy
  \item 3.58 -- Naming Ions
\end{itemize}

\section{Polyatomic Ions}
\begin{itemize}
	\item Polyatomic ions are composed of more than one atom
	\item These polyatomic ions are a covalently bound group, and should be considered as a single ion
	\item See table 3.3
	\item There are many series of polyatomic ions which differ in only the number of oxygen atoms
	\item There are also series of polyatomic ions which differ in hydrogen atoms and charge (each \ch{H} increases the charge by $1$)
\end{itemize}

\section{Ionic Bonds}
\begin{itemize}
	\item Ionic compounds are held together by ionic bonds
	\item Positive cations are attracted to negative anions
	\item Ionic solids exhibit a regular crystal lattice on atomic scales
\end{itemize}

\section{Formulas of Ionic Compounds}
\begin{itemize}
	\item To find the formula for an ionic compound, we need to balance the charges to give a neutral compound
	\item The ratio should be in a mathematically reduced form, making just one \emph{formula unit}
\end{itemize}

\section{Naming Ionic Compounds}
\begin{itemize}
	\item Simply combine the two ion names -- Cation first, then anion
	\item Do not state the ratio between them, like you do in molecular compounds
	\item Table 3.4 -- Some common ionic compounds and their applications
\end{itemize}

\section{Some Properties of Ionic Compounds}
\begin{itemize}
	\item Ionic compounds have different lattice structures, depending on the sizes and charges of the ions
	\item Ionic solids are rigid but brittle, easily cleaved along crystal planes
	\item Ionic compounds have high melting and boiling points -- This is because ionic bonds are very strong
	\item Some ionic compounds dissolve in water by making many attractive interactions with water dipoles. Others are just bound too tightly, and are insoluble.
\end{itemize}

\section{\ch{H^+} and \ch{OH^-} Ions: An Introduction to Acids and Bases}
\begin{itemize}
	\item The hydrogen cation (really just a bare proton) and hydroxide anion (\ch{OH^-}) are very important to aqueous chemistry
	\item When an acid dissolves in water, it donates a hydrogen cation to a water molecule forming \ch{H3O^+}
	\item When a base dissolves in water, it will accept a proton from water forming \ch{OH^-}
	\item Some acids can provide multiple \ch{H^+} in reactions (\ch{HCl}, \ch{H2SO3}, \ch{H3PO4}), and some bases can accept multiple protons (\ch{NaOH}, \ch{Ba(OH)2})
\end{itemize}

\section*{Homework 3.2:}
\begin{itemize}
  \item 3.62 -- Polyatomic Ions
  \item 3.66 -- Ions combine to form compounds
  \item 3.68 -- Formulas from names for ionic compounds
  \item 3.70 -- Names from formulas for ionic compounds
  \item 3.76 -- Acid/base reactions with water
\end{itemize}

\chapter{Molecular Compounds}
\section{Covalent Bonds}
\begin{itemize}
	\item A bond formed by the \emph{sharing} of electrons is called a covalent bond. They occur between non-metals
	\item A group of atoms held together by covalent bonds is called a molecule (this can include compounds like \ch{CO2} or elements like \ch{O2})
	\item Show a water molecule using dot diagrams
	\item Repulsive and attractive forces in a covalent bond as a function of bond length
	\item Overlap of 1s orbitals for \ch{H2} and 2p orbitals for \ch{F2}
\end{itemize}

\section{Covalent Bonds and the Periodic Table}
\begin{itemize}
	\item The periodic table can be used to predict how many bonds an element tends to make
	\item Consider \ch{HF}, \ch{H2O}, \ch{NH3}, and \ch{CH4}
	\item This pattern is the result of the octet rule. An atom makes as many bonds as it needs in order to reach a full octet
	\item Elements in the third row and below can form more bonds. See Figure 4.3
\end{itemize}

\section{Multiple Covalent Bonds}
\begin{itemize}
	\item Some molecules cannot satisfy the octet rule using only single bonds
	\item Consider \ch{F2}, \ch{O2}, and \ch{N2}
	\item Single, double, and triple bonds share 2, 4, and 6 electrons
	\item Multiple bonds in organic molecules
\end{itemize}

\section{Coordinate Covalent Bonds}
\begin{itemize}
	\item Coordinate covalent bonds are when both shared electrons came from one atom in the pair
	\item Identify them in molecular compounds by counting the electrons and comparing them to their valency: \ch{N2O}
	\item \ch{O3} and \ch{NH3BF3} are two more examples of coordinate covalent compounds
	\item In acid/base chemistry, all acidic hydrogens form coordinate covalent bonds
\end{itemize}

\section*{Homework 4.1:}
\begin{itemize}
  \item 4.34 -- Identify covalent and ionic bonds
  \item 4.36 -- Predict \# of bonds from periodic table
  \item 4.42 -- Identify coordinate covalent bonds
\end{itemize}

\section{Characteristics of Molecular Compounds}
\begin{itemize}
	\item Molecular compounds have a wide range of melting and boiling points, depending on their intermolecular forces (chapter 8)
	\item Few molecular compounds are soluble in water, but many are soluble in organic solvents
	\item Most molecular compounds do not conduct electricity, either as a pure substance or dissolved in water
\end{itemize}

\section{Molecular Formulas and Lewis Structures}
\begin{itemize}
	\item Molecular formulas tell the types and numbers of atoms found in a molecular compound
	\item Structural formulas use lines to represent covalent bonds and show the structure of a molecule
	\item Lewis structures add dots to represent lone pairs of electrons
\end{itemize}

\section{Drawing Lewis Structures}
\begin{itemize}
	\item The first way to draw Lewis structures works for most simple organic molecules
	      \begin{itemize}
		      \item If the only atoms are C, H, O, X (Halogen), and H, then this approach might work
		      \item Arrange the carbon atoms in a chain, or backbone
		      \item Attach the heteroatoms with the appropriate number of bonds (based on the octet rule)
		      \item Make any multiple bonds in the carbon chain needed to complete their octets
		      \item If all of the elements have a full octet, and form the correct number of bonds, then it is a valid structure
		      \item Starting from a condensed structure makes this method particularly easy
	      \end{itemize}
	\item The next method is very useful for molecular compounds with more complex structure, or central atoms which aren't carbon
	      \begin{enumerate}
		      \item Count up the total number of valence electrons (taking into account any net charge)
		      \item Draw single bonds from the central atom to all peripheral atoms
		      \item Fill the octets of the outer atoms with lone-pairs (H doesn't need any)
		      \item Place any remaining electrons onto the central atom
		      \item Satisfy the octet rule for the central atom, if needed, by converting outer lone-pairs into multiple-bonds with the central atom
	      \end{enumerate}
	\item Practice: \ch{PCl3}, \ch{HCN}, \ch{SO4^{2-}} (sulfate has single bonds for this class)
\end{itemize}

\section{The Shapes of Molecules}
\begin{itemize}
	\item We know that molecules have a particular shape, defined by VSEPR theory
	\item We can predict the molecular geometry based on a molecule's Lewis structure
	      \begin{enumerate}
		      \item Draw a valid Lewis structure
		      \item Count the number of electron domains
		      \item Count the number of bonds vs. lone pairs
		      \item Consult Table 4.2 (or your memory)
	      \end{enumerate}
	\item Bond angles can range from $180^\circ$ to $<109.5^\circ$
	\item If lone pairs are present, they slightly reduce the bonding angles
\end{itemize}

\section*{Homework 4.2:}
\begin{itemize}
  \item 4.53 -- Condensed structural formulas
  \item 4.56 -- Drawing Neutral Lewis structures
  \item 4.62 -- Drawing Lewis Structures of polyatomic ions
  \item 4.66 -- Molecular geometry
\end{itemize}

\section{Polar Covalent Bonds and Electronegativity}
\begin{itemize}
	\item Any covalent bond between different atoms will share electrons unequally
	\item This unequal sharing leads to a polar bond
	\item The direction and intensity of the bond polarity is determined by comparing electronegativities of the two elements
	\item F is the most electronegative, while Rb is the least
	\item Electronegativity differences of 0-0.4 are considered normal covalent bonds
	\item Electronegativity differences of 0.5-1.9 are considered polar covalent bonds
	\item Electronegativity differences greater than 2 are considered ionic bonds
\end{itemize}

\section{Polar Molecules}
\begin{itemize}
	\item Polar bonds in a molecule can lead to a charge dipole over the entire molecule
	\item First find the polar bonds, then add up their dipole vectors to find the net dipole
	\item Symmetry can often lead to a non-polar molecule despite having polar bonds
\end{itemize}

\section{Naming Binary Molecular Compounds}
\begin{itemize}
	\item A compound made of only two types of atoms is called a binary compound
	\item When naming a binary compound, the less electronegative atom usually comes first
	\item The second element will have an ``-ide'' ending, just like in ionic compounds
	\item Use prefixes (Table 4.3) to indicate how many of each element are present
\end{itemize}

\section*{Homework 4.3:}
\begin{itemize}
  \item 4.74 -- Polar bonds
  \item 4.76 -- Polar bonds in molecules
  \item 4.78 -- Polarity of molecules
  \item 4.82 -- Names from formulas
  \item 4.84 -- Formulas from names
\end{itemize}

\chapter{Classification and Balancing of Chemical Reactions}
\section{Chemical Equations}
\begin{itemize}
	\item Chemical equations can be thought of as recipes which show what you start with and what you get out
	\item Coefficients indicate how many units of each reactant are needed, and how many units of each product are made
  \item These ratios are referred to by the name \emph{stoichiometry}
	\item We find the coefficients by relying on the law of conservation of mass (balancing)
\end{itemize}

\section{Balancing Chemical Equations}
\begin{itemize}
	\item To balance chemical reactions:
  \begin{itemize}
    \item Write the unbalanced equation, with correct formulas for all reactants and products
    \item Identify an element which is unbalanced, and balance it with a coefficient
    \item Repeat the above step until all elements are balanced
    \item Reduce the coefficients if they share a common factor
  \end{itemize}
  \item Combustion of \ch{C5H10} gives an example where we have to double the coefficients
\end{itemize}

\section*{Homework 5.1:}
\begin{itemize}
  \item 5.24 -- Balance equations from chemical names
  \item 5.28 -- Balance general equations
  \item 5.30 -- Balance combustion reactions
\end{itemize}

\section*{Classes of Chemical Reactions}
\begin{itemize}
  \item This section was removed from the new edition, but it is worth providing an overview of the classes of reactions before we go over each in detail
	\item It is sometimes helpful to recognize different classes of chemical reactions
	\item Precipitation reactions go from aqueous ionic reactants to solid ionic products:

	      \ch{Pb(NO3)2(aq) + 2 KI(aq) -> 2 KNO3(aq) + PbI2(s)}
      \item Acid-base neutralization reactions result in water and a salt (\ch{H+} is transferred)

	      \ch{HCl(aq) + NaOH(aq) -> H2O(l) + NaCl(aq)}
	\item Oxidation-reduction reactions change the oxidation state (or charge) of atoms:

	      \ch{Mg(s)+I2(g)->MgI2(s)}
\end{itemize}

\section{Precipitation Reactions and Solubility}
\begin{itemize}
	\item Solubility is a measure of how much a substance will dissolve in a solvent (usually water)
	\item In precipitation reactions, you must determine the solubility of \emph{all} potential products
	\item Solubility rules (Table 5.1):
	      \begin{itemize}
		      \item A compound is probably soluble if its cation is ammonium (\ch{NH4^+}) or group 1A cations (\ch{Li^+}, \ch{Na^+}, \ch{K^+}, \ch{Rb^+}, or \ch{Cs^+})
		      \item A compound is probably soluble if it contains one of the following anions:

		            Halide ions (except when paired with \ch{Ag^+}, \ch{Hg2^{2+}}, or \ch{Pb^{2+}} ions)

		            Nitrate, perchlorate, acetate, or sulfate (except for sulfates paired with \ch{Ba^{2+}}, \ch{Hg2^{2+}}, or \ch{Pb^{2+}} ions)
		      \item Most other ionic compounds are not soluble
		      \item Net ionic equation omit all species which are aqueous on both sides of the reaction
	      \end{itemize}
\end{itemize}

\section*{Net Ionic Equations}
\begin{itemize}
  \item We are covering this topic early because of its relevance to precipitation reactions
  \item Aqueous ionic compounds have cations and anions which are not actually bound together at all
  \item Separate aqueous ionic compounds into their ions -- this is the complete ionic equation
  \item Identify spectators -- species which are in exactly the same form on both the reactant and product sides
  \item Eliminate the spectators, leaving only the species which are involved in real chemistry -- This is the net ionic equation
\end{itemize}

\section*{Homework 5.2:}
\begin{itemize}
  \item 5.40 -- Ion solubility rules
  \item 5.42 -- Predicting solubility of reaction products
  \item 5.46 -- Writing net ionic equations
\end{itemize}

\section{Acids, Bases, and Neutralization Reactions}
\begin{itemize}
	\item When a strong acid and base react, the product is \emph{neutral} (neither an acid nor a base)
	\item The products can include water and a neutral ionic compound
	\item Some acid/base neutralization reactions produce gas: \ch{2 HCl(aq) + K2CO3(aq) -> H2O(l) + 2 KCl(aq) + CO2(g)}
	\item Many bases yield \ch{OH^-} as a product, even if they dont have hydroxide as a component themselves: \ch{NH3 + H2O -> NH4^+ + OH^-}
\end{itemize}

\section{Redox Reactions}
\begin{itemize}
	\item Redox reactions involve changes in oxidation state, or the exchange of electrons
	\item OIL-RIG -- ``Oxidation is losing, Reduction is gaining'' electrons
	\item Oxidation leads to a higher oxidation state, while reduction leads to a lower (reduced) oxidation state
	\item Reduction and Oxidation always come together, since the electrons need a source and a destination
	\item We use language which can be confusing at first:
	      \begin{itemize}
		      \item The oxidizing agent is the chemical which oxidizes its partner. It is reduced itself in the process. It is a destination for electrons
		      \item The reducing agent is the chemical which reduces its partner. It is oxidized itself in the process. It is a source for electrons
	      \end{itemize}
	\item Some redox processes we encounter in the world around us: Corrosion, Combustion, Respiration, Bleaching, Metallurgy
\end{itemize}

\section{Recognizing Redox Reactions}
\begin{itemize}
	\item Determining oxidation states can get complicated, but for now we can simply relate oxidation state to charge
	\item Elemental states have oxidation states of 0
	\item For a covalently bonded compound, the oxidation states sum up to the total charge
	\item Elements in a covalent bond will usually take the same oxidation state that they would in an ionic bond (C is an important exception, as well as non-metal oxides)
\end{itemize}

\section{Net Ionic Equations}
\begin{itemize}
  \item Covered out of order above
\end{itemize}

\section*{Homework 5.3:}
\begin{itemize}
  \item 5.34 -- Classify reactions by type
  \item 5.52 -- Assigning oxidation numbers
  \item 5.54 -- Recognize oxidation and reduction
  \item 5.58 -- Oxidizing and reducing agents
\end{itemize}

\chapter{Chemical Reactions: Mole and Mass Relationships}
\section{The Mole and Avogadro's Number}
\begin{itemize}
	\item The molecular weight or formula weight of a substance is the sum of atomic weights for its constituent atoms
	\item Different substances have different molecular weights, so $1.0g$ of \ch{C2H4} has more molecules in it than $1.0g$ of \ch{HCl}
	\item If we wanted to make \ch{C2H5Cl} without any left-over reactants, then we would need more grams of \ch{HCl} than of \ch{C2H4}
	\item Find the ratio by comparing the formula masses ($28.0amu$ vs $36.5amu$). If I have $1.0g$ of \ch{C2H4}, how many grams of \ch{HCl} do I need? ($1.3g$)
	\item The mole is an amount that relates amus (atomic scale) to grams (macroscopic scale)
	\item The molar mass tells how many grams of a substance is one mole of molecules
	\item $N_A=6.022\times10^{23}$, Avogadro's number, is the number of amus per gram
\end{itemize}

\section{Gram-Mole Conversions}
\begin{itemize}
	\item We can convert from grams to moles and vice-versa using the molar mass
	\item How many moles is $2.5g$ of \ch{C2H5Cl}? How many molecules?
\end{itemize}

\section{Mole Relationships and Chemical Equations}
\begin{itemize}
	\item A balanced chemical equation relates the ratios of moles of reactants and products
	\item The coefficients tell how many moles are needed or produced
	\item We call these ratios \emph{stoichiometry}
\end{itemize}

\section*{Homework 6.1:}
\begin{itemize}
  \item 6.30 -- Grams to moles
  \item 6.32 -- Calculating molar mass
  \item 6.34 -- Moles to grams
  \item 6.36 -- Stoichiometric ratios in a balanced equation
\end{itemize}

\section{Mass Relationships and Chemical Equations}
\begin{itemize}
	\item Mole to mole conversions rely on stoichiometric ratios (coefficients in a balanced chemical equation)
	\item Mole to mass and mass to mole conversions rely on molar mass
	\item Mass to mass conversions must follow the path: \ch{mass1 -> moles1 -> moles2 -> mass2}
	\item You should always check your calculated answer against a rough estimate to make sure it makes sense
	\item For \ch{H2SO4(aq) + 2 NaOH(aq) -> Na2SO4(aq) + 2 H2O(aq)}:
	      \begin{itemize}
		      \item How many  moles of \ch{NaOH} needed to react with $0.490g$ \ch{H2SO4}?
		      \item How many grams of water and sodium sulfate would be produced?
	      \end{itemize}
\end{itemize}

\section{Limiting Reagent and Percent Yield}
\begin{itemize}
	\item Unless the amounts of reactants are carefully chosen, one will run out first and the other will have left-over remaining after the reaction
	\item To find the limiting reagent, calculate how much product each one would yield. The reactant which yields less is the limiting reagent and the amount calculated is the theoretical yield
	\item To find percent yield, you must have an actual mass of product:

	      $\% Yield = \dfrac{Actual Yield}{Theoretical Yield}\times 100\%$
	\item \ch{CO2 + H2O -> C6H12O6 + O2} (unbalanced)

	      With $10.0g$ of \ch{CO2} and $5.00g$ of \ch{H2O}, how many grams of sugar are produced?
	\item if $4.8g$ of sugar are actually recovered, what is the \% recovery?


	      With $1.0g$ citric acid and $1.0g$ sodium bicarbonate how many grams of \ch{Na3C6H5O7} are produced?

      $M=192.12\nicefrac{g}{mol}$ ~ $M=84.01\nicefrac{g}{mol}$ ~ $M=258.07\nicefrac{g}{mol}$ ~ $M=18.015\nicefrac{g}{mol}$ ~ $M=44.01\nicefrac{g}{mol}$





	\item \ch{C6H8O7 + NaHCO3 -> Na3C6H5O7 + H2O + CO2} (unbalanced)

      ($0.238~g$) \hspace{0.2em} ($0~g$) \hspace{3em} ($1.024~g$) \hspace{1.5em} ($0.214~g$) \hspace{1.5em} ($0.524~g$)

	\item You actually recover $0.56g$ of citrate. What is the \% yield? ($54.7\%$)
\end{itemize}

\section*{Homework 6.2:}
\begin{itemize}
  \item 6.40 -- Finding required co-reactant mass
  \item 6.50 -- Limiting reactant problem
\end{itemize}

\chapter{Chemical Reactions: Energy, Rates, and Equilibrium}
Some chemical reactions can be balanced, but meaningless because they will never occur naturally.

\ch{2 Au(s) + 3 H2O(l) -> Au2O3(s) + 3 H2(g)}

\ch{C_{(diamond)} -> C_{(graphite)}}

\section{Energy and Chemical Bonds}
\begin{itemize}
	\item Potential energy is stored energy. The energy of chemical bonds is a form of negative potential energy
	\item Kinetic energy is energy of motion. Motion on the atomic scale is observed as heat
	\item In a chemical reaction, energy is required to break chemical bonds, but energy is released when new bonds are formed
	\item If the energy released is greater than the energy required to break the bonds, then the reaction produces heat and the products are more stable than the reactants
\end{itemize}

\section{Heat Changes During Chemical Reactions}
\begin{itemize}
	\item To predict if a reaction will give off heat, we must compare the energies of the bonds broken to the energies of the bonds formed
	\item Table 7.1 shows average bond enthalpies
	\item Consider the reaction: \ch{H2(g) + Cl2(g) -> 2 HCl} -- $\Delta H_{rxn}\approx-43\dfrac{kcal}{mol}$
	\item Since energy is conserved and the molecules lost energy overall, then that energy must have gone somewhere -- It is released as heat!
	\item Reactions which release heat are called ``exothermic,'' and they occur more readily in nature
	\item Reactions which take in heat (make the surroundings colder) are called ``endothermic,'' and they are a bit harder to find in nature
	\item Demo -- Burning methanol and \ch{Ba(OH)2$\cdot$8 H2O(s) + 2 NH4SCN(s) -> Ba(SCN)2(aq) + 2 NH3(g) + 10 H2O(l)}
	\item We have a fancy name for the heat of reaction carried under constant pressure conditions: ``Enthalpy of Reaction'' ($\Delta H$)
\end{itemize}

\section{Exothermic and Endothermic Reactions}
\begin{itemize}
	\item Practice an exothermic reaction: \ch{CH4(g) + 2 O2(g) -> CO2(g) + 2 H2O(l)} \hspace{1em} $\Delta H_{rxn}=-196\dfrac{kcal}{mol}$
	\item Practice an endothermic reaction: \ch{N2(g) + O2(g) -> 2 NO(g)} \hspace{1em} $\Delta H_{rxn}=43\dfrac{kcal}{mol}$
	\item Note that these quantities are -per mole. Table 7.2 compares combustion enthalpies on a -per gram basis
\end{itemize}

\section{Why do Chemical Reactions Occur? Free Energy}
\begin{itemize}
	\item Chemical reactions which occur in nature are called ``spontaneous''
	\item Non-spontaneous reactions can be made to occur if driven by some external input of energy
	\item Enthalpy is one governing factor in spontaneity, and entropy ($S$) is the other
	\item Entropy can be understood as the amount of disorder in a system
	\item A reaction will have $\Delta S_{rxn}>0$ if the products are more disordered than the reactants
	\item Entropy and Enthalpy are combined to form ``free energy'' ($\Delta G_{rxn}$), which can be thought of as a measure of the spontaneity of a reaction
	\item $\Delta G_{rxn} = \Delta H_{rxn} - T\Delta S_{rxn}$
	\item $\Delta H_{rxn}<0$ and $\Delta S_{rxn}>0$ both trend toward spontaneity -- Draw the quadrant diagram
  \item When is the following spontaneous? \ch{3 H2(g) + N2(g) <=> 2 NH3(g)} \hspace{1em} $\Delta H_{rxn}=-92\nicefrac{kJ}{mol}$
\end{itemize}

\section{How do Chemical Reactions Occur? Reaction Rates}
\begin{itemize}
	\item $\Delta G_{rxn}$ and free energy only tell us if a reaction can happen, but doesn't say how quickly a reaction will occur
	\item Some spontaneous reactions happen so slowly that they never really happen at all
	\item The rate of a reaction depends on different factors, like how the reaction actually occurs (from a molecular perspective)
	\item For reactions to occur, molecules must collide with enough energy and in the correct orientation (Figure 7.2)
	\item The energy required to start a reaction is called the activation energy, and it limits the rate of reaction
	\item Reaction coordinate diagrams can show factors which control both the energetics and the rate of a reaction(Figure 7.3)
\end{itemize}

\section*{Homework 7.1:}
\begin{itemize}
  \item 7.28 -- Reaction enthalpies from bonds
  \item 7.32 -- Predicting entropy changes
  \item 7.38 -- Temperature and spontaneity
\end{itemize}

\section{Effects of Temperature, Concentration, and Catalysts on Reaction Rates}
\begin{itemize}
	\item Temperature:
	      \begin{itemize}
		      \item Increasing the temperature increases the frequency and energy of molecular collisions
		      \item Both of these factors will increase the rate of any reaction
	      \end{itemize}
	\item Concentration:
	      \begin{itemize}
		      \item Increasing the concentration increases the frequency of molecular collisions
		      \item This will increase the rate of any reaction
	      \end{itemize}
	\item Catalysts:
	      \begin{itemize}
		      \item Catalysts change the way a reaction proceeds, requiring less activation energy (Figure 7.4)
		      \item This will increase the rate of reaction, since more collisions will have the necessary energy
	      \end{itemize}
\end{itemize}

\section*{Homework 7.2:}
\begin{itemize}
  \item 7.21 -- Reaction coordinate diagrams
  \item 7.44 -- Concentration and reaction rates
  \item 7.46 -- Catalysts and activation energy
\end{itemize}

\section{Reversible Reactions and Chemical Equilibrium}
\begin{itemize}
	\item Some chemical reactions can go in the forward or reverse direction dependig on the circumstances
	\item These reactions are called equilibrium reactions because they establish an equilibrium state where both reactants and products are present
	\item At equilibrium the forward and reverse reactions occur at the same rate, so the concentrations of products and reactants remains constant.
\end{itemize}

\section{Equilibrium Equations and Equilibrium Constants}
\begin{itemize}
	\item We can describe the equilibrium conditions for a reaction using math
	\item The equilibrium constant: $K_{eq}=\dfrac{[M]^m[N]^n}{[A]^a[B]^b}$
	\item The exponents in $K_{eq}$ are from the coefficients in the balanced chemical equation
	\item We can therefore find the equilbrium constant if we measure the equilibrium amounts of reactants and products
	\item We could also find the equilbrium amount of one species if we know $K_{eq}$ and the concentrations of all other species
  \item Consider the reaction \ch{H2C2O4(aq) + 2 H2O(l) <=> 2 H3O^{+}(aq) + C2O4^{2-}(aq)}

    Find $K_{eq}$ if $[\ch{H2C2O4}]=0.250M$, $[\ch{H3O^+}]=1.0\times10^{-3}M$, and $[\ch{C2O4^{2-}}]=0.750M$ \hspace{1em} ($2.9\times10^{-6}$)

    Find $[\ch{H3O^+}]$ if $[\ch{H2C2O4}]=0.050M$ and $[\ch{C2O4^{2-}}]=1.50M$ \hspace{1em} ($3.1\times10^{-14}$)
	\item The value of $K_{eq}$ can indicate whether thee are more products or reactants at equilibrium
	\item $K_{eq}>1$ favors products, while $K_{eq}<1$ favors reactants
\end{itemize}

\section{Le Ch\^atelier's Principle: The Effect of Changing Conditions on Equilibria}
\begin{itemize}
	\item Le Ch\^atelier's principle describes how a system will respond after being perturbed away from equilibrium
	\item Effect of Changes in Concentration:
	      \begin{itemize}
		      \item When concentrations are changed, the system will respond to counteract the change
		      \item Removing product will shift the reaction toward products
		      \item Adding product will shift the reaction toward reactants
		      \item etc.
	      \end{itemize}
	\item Effect of Changes in Temperature
	      \begin{itemize}
		      \item To understand the effect of a change in temperature we can consider heat as either a reactant or a product
		      \item Heat acts as a product in an exothermic reaction, and as a reactant in an endothermic reaction
		      \item Increasing the temperature is like adding heat, and decreasing the temperature is like removing heat
		      \item When temperature changes, the value of $K_{eq}$ actually changes
	      \end{itemize}
	\item Effect of Changes in Pressure (or volume):
	      \begin{itemize}
		      \item First, count the number of moles on each side of the reaction (gas for pressure and aqueous for volume)
		      \item Increasing pressure or decreasing volume will shift the reaction toward the side with fewer moles
		      \item Decreasing the pressure increasing the volume will shift the reaction toward the side with more moles
	      \end{itemize}
\end{itemize}

\section*{Homework 7.3:}
\begin{itemize}
  \item 7.54 -- Equilibrium expressions and $K_{eq}$
  \item 7.56 -- Finding equilibrium concentrations
  \item 7.66 -- LeCh\^atelier's principle
\end{itemize}

\chapter{Gases, Liquids, and Solids}
\section{The States of Matter and Their Changes}
\begin{itemize}
	\item The three primary phases of matter: Solid, Liquid, and Gas
	\item Phase changes can go either direction at the phase change temperature
	\item \ch{s->l} and \ch{l->g} are endothermic, while the reverse reactions are exothermic
	\item Phase change temperatures can be understood in the context of: $\Delta G = \Delta H - T\Delta S$
	\item Sublimation and dry ice
\end{itemize}

\section{Intermolecular Forces}
\begin{itemize}
	\item Intermolecular forces determine the physical phase of a substance
	\item London Dispersion Forces
	      \begin{itemize}
		      \item Figure 8.4
		      \item Effect of molecular shape -- More surface area gives stronger forces
	      \end{itemize}
	\item Dipole-Dipole Forces
	      \begin{itemize}
		      \item Figure 8.3
		      \item Effect of polarity -- Stronger dipoles give stronger forces
	      \end{itemize}
	\item Hydrogen Bonds
	      \begin{itemize}
		      \item H-bonds special cases which go beyond normal dipole-dipole bonds
		      \item A H atom is very strongly attracted to another molecule (almost like the H is shared between molecules)
		      \item Page 221 un-numbered figure
		      \item Electron donor must be an electronegative atom (N, O, or F)
		      \item Hydrogen must be bound to an electronegative atom (N, O, or F)
		      \item Acetone is a counter-example. It has O and H, but not bound together
	      \end{itemize}
\end{itemize}

\section{Gases and the Kinetic-Molecular Theory}
\begin{itemize}
	\item Gas behavior can be mostly explained by assuming that the individual gas molecules follow these assumptions:
	      \begin{itemize}
		      \item A gas consists of many particles which move randomly and independently
		      \item The individual gas particles exhibit no intermolecular forces
		      \item The volume occupied by the gas particles is vanishingly small compared to the volume filled by the gas collectively (i.e. the particles are very small)
		      \item The average kinetic energy of the gas particles is proportional to the temperature (in K)
		      \item Gas particles collide elastically
	      \end{itemize}
	\item Real gases deviate slightly from these assumptions, but we can ignore those deviations and treat them like an \emph{ideal gas}
\end{itemize}

\section{Pressure}
\begin{itemize}
	\item Pressure is the force a gas applies over an area
	\item Even ambient air is exerting a fairly high pressure -- atmospheric pressure
	\item We measure ambient pressure using a mercury barometer -- Figure 8.9
	\item Pressure can be measured in many different units: $1atm=760mmHg=760torr=14.7psi=101325Pa$
\end{itemize}

\section{Boyle's Law}
\begin{itemize}
	\item Early physicists experimented with the fundamental state variables of gases ($P$, $V$, and $T$) and noticed simple relationships called gas laws
	\item Boyle showed the relationship between pressure and volume
	\item Demo a small balloon expanding in a vacuum: $P\propto \dfrac{1}{V}$
	\item The pressure increases with decreased volume, which is consistent with the kinetic molecular theory of gases because there are more molecular collisions per second in the smaller volume
	\item Solve Boyle's law problems by relating the initial and final conditions: $P_1V_1=P_2V_2$
\end{itemize}

\section{Charles's Law}
\begin{itemize}
	\item Charles's law relates the volume of a gas to temperature: $V\propto T$
	\item Demo a small balloon expanding as it moves from ice to boiling water
	\item This behavior is explained by the kinetic molecular theory of gases because the collisions are more forceful, pushing to greater volumes
	\item Solve Charles's law problems with the following relation: $\dfrac{V_1}{T_1}=\dfrac{V_2}{T_2}$
\end{itemize}

\section*{Homework 8.1:}
\begin{itemize}
  \item 8.34 -- Intermolecular forces
  \item 8.42 -- Converting gas pressures
  \item 8.48 -- Boyle's Law
  \item 8.56 -- Charles's Law
\end{itemize}

\section{Gay-Lussac's Law}
\begin{itemize}
	\item Gay-Lussac's law provides the last relation: $P\propto T$
	\item This law is redundant, since the previous two laws could be combined to prove the Gay-Lussac law
\end{itemize}

\section{The Combined Gas Law}
\begin{itemize}
	\item Boyle's law, Charles's law, and the Gay-Lussac law all focus on the relationship between two state variables while keeping the third constant
	\item These laws can be combined to give a law which allows all three variables to change

	      $\dfrac{PV}{T} = k$ where $k$ is a constant value
	\item Solve these problems with the following relation: $\dfrac{P_1V_1}{T_1}=\dfrac{P_2V_2}{T_2}$
	\item This law includes all the others, so it is really the only one that you need to know, except for the even better law we will teach you next class
\end{itemize}

\section{Avogadro's Law}
\begin{itemize}
	\item All of the above laws are for a constant (and unknown) number of moles of gas
	\item More moles of gas will give greater volume : $\dfrac{V_1}{n_1} = \dfrac{V_2}{n_2}$
	\item It is convenient to define a standard temperature ($0^\circ C$) and pressure ($1atm$) or comparing different gas systems
	\item At STP, one mole of an ideal gas occupies a standard molar volume of $22.4L$
\end{itemize}

\section{The Ideal Gas Law}
\begin{itemize}
	\item The ideal gas law combines everything we have covered so far: $PV=nRT$
	\item $R$ is the gas constant, which can take on different values depending on the units:

	      $R=0.08206\dfrac{L~atm}{mol~K} = 62.4\dfrac{L~Torr}{mol~K}$
	\item Find the pressure when $0.250mol$ of gas are placed in a $5.20L$ container at $15.0^\circ C$ ($1.14atm$)
	\item How many moles are in a $3.15L$ helium balloon at $20.0^\circ C$ and $0.820atm$? ($0.107mol$)
\end{itemize}

\section{Partial Pressures and Dalton's Law}
\begin{itemize}
	\item Since gas particles don't interact at all, a mixture of gases doesn't introduce any new complexity
	\item Each gas exerts a partial pressure as if the other gases weren't there, and the total pressure is the sum of the partial pressures
	\item $P_{total} = \sum P_{i, partial}$
	\item The air is mostly made of \ch{O2} and \ch{N2}. In Cedar City, $P_{\ch{O2}}=0.17atm$ and $P_{\ch{N2}}=0.64atm$. What is the total barometric pressure? ($0.81atm$)
\end{itemize}

\section*{Homework 8.2:}
\begin{itemize}
  \item 8.62 -- Ideal gas law
  \item 8.82 -- Ideal gas law
  \item 8.88 -- Dalton's law of partial pressures
\end{itemize}

\section{Liquids}
\begin{itemize}
	\item A liquid in a closed container will reach equilibrium with the vapor phase as vaporization and condensation rates come into balance (Figure 8.19)
	\item The equilibrium pressure above a liquid is the vapor pressure
	\item Different liquids have different vapor pressures, and vapor pressure always increases with temperature (Figure 8.20)
	\item When the vapor pressure equals the external pressure ($\approx 1atm$), boiling will begin
	\item The lower external pressure at high altitudes is why water will boil at less than $100^\circ C$ at those altitudes
	\item Viscosity is a measure of how freely a liquid flows. Some liquids, like honey, are very viscous
	\item Surface Tension is the resistance of a liquid to spreading out and increasing its surface area. This leads to curved droplets of liquids, since this shape minimizes surface area (Figure 8.21)
\end{itemize}

\section*{Water: A Unique Liquid}
\begin{itemize}
	\item Water has many unique properties compared to other liquids
	\item Water has an unusually high heat capacity and enthalpy of vaporization due to its strong hydrogen-bonding intermolecular forces
	\item Water is nearly unique in having a solid phase which is less dense than the liquid phase. This is because ice has an unusually open crystal structure (Figure 8.23)
\end{itemize}

\section{Solids}
\begin{itemize}
	\item Many solids exhibit an ordered structure over long ranges. These are called crystalline solids
	\item Crystalline solids can be one of four types:
	      \begin{itemize}
		      \item Ionic solids, like \ch{NaCl}, are made of a lattice of cations and anions
		      \item Molecular solids, like sugar, are made of individual molecules arranged in a repeating lattice
		      \item Covalent network solids, like diamond, are extended networks of covalent bonds. A diamond can \emph{kind of} be described as one giant molecule!
		      \item Metallic solids are made of a lattice of metal atoms which share a sea of electrons across the whole solid. This is why metals conduct electricity
	      \end{itemize}
	\item Amorphous solids, like glass, show no long-range structure. They are formed when liquids cool too quickly for a proper lattice to form
	\item Amorphous solids slowly grow softer over a wide temperature range, rather than melting sharply at a single melting temperature
\end{itemize}

\section{Changes of State}
\begin{itemize}
	\item When a substance undergoes a phase change, it will either give off or take in heat without changing the temperature
	\item The amount of heat is called the heat of fusion or heat of vaporization
	\item Figure 8.24 shows a heating curve across two phase changes
	\item Heat for changing temperature: $q=mC_p\Delta T$
	\item Heat for a phase change: $q=m\Delta H_{f or v}$
	\item $7.5g$ of ice begin at $-6.00^\circ C$. Find how many $J$ are required to convert this ice into steam at $125^\circ C$ ($23,000J$)

	      For water: $C_{s}=2.11\dfrac{J}{g~^\circ C}$ \hspace{1em} $C_{l}=4.18\dfrac{J}{g~^\circ C}$ \hspace{1em} $C_{g}=2.00\dfrac{J}{g~^\circ C}$ \\ \hphantom{For Water:} $\Delta H_{fus} = 334\dfrac{J}{g}$ \hspace{1em} $\Delta H_{vap} = 2260\dfrac{J}{g}$
\end{itemize}

\section*{Homework 8.3:}
\begin{itemize}
  \item 8.92 -- Pressure and boiling points
  \item 8.96 -- Amorphous and crystalline solids
  \item 8.98 -- Enthalpy of fusion problem
\end{itemize}

\chapter{Solutions}
\section{Mixtures and Solutions}
\begin{itemize}
	\item Recall the taxonomy of mixtures from chapter 1: Heterogeneous mixtures vs homogeneous mixtures, and solutions vs colloids
	\item The primary difference is particle size: Solutions have particles $<2nm$, colloids have particles $2-500nm$, and heterogeneous mixtures have particles $>500nm$
	\item In a solution, the solvent is the dominant component while any minority components are solutes
	\item Solutions can be more than just solid-in-liquid -- Table 9.2
\end{itemize}

\section{The Solvation Process}
\begin{itemize}
	\item In solvation, solute-solute interactions and solvent-solvent interactions are replaced by solute-solvent interactions
	\item To determine solubility, consider the various interactions
	      \begin{itemize}
		      \item Stronger solute-solute interactions will make it hard to break apart the solute
		      \item Stronger solvent-solvent interactions will make it hard to disrupt solvent to make room for solute
		      \item Similar solute-solute, solvent-solvent, and solute-solvent interactions lead to high solubility
		      \item This leads to the rule of thumb that ``like dissolves like''
	      \end{itemize}
	\item The solvent will form a solvation sphere around the solute
	\item The net result can be exothermic or endothermic
\end{itemize}

\section*{Solid Hydrates}
\begin{itemize}
	\item Some ionic compounds will naturally incorporate water molecules in their ionic lattice
	\item These compounds are called ``hygroscopic,'' and are written with a dot like this: \ch{CaSO4}$\cdot \dfrac{1}{2}$ \ch{H2O}
\end{itemize}

\section{Solubility}
\begin{itemize}
	\item Solubility is the amount of one substance that you can dissolve in another substance
	\item If two liquids can be mixed in any ratio, then they are said to be miscible
	\item If a solution has dissolved as much of a solid as it can, then it is called ``saturated''
	\item A saturated solution can be identified if excess solute remains undissolved
\end{itemize}

\section{Temperature Dependence of Solubility}
\begin{itemize}
	\item Increasing the temperature will increase the solubility of most, but not all solids
	\item If solvation is endothermic, then solubility increases with increasing temperature. If solvation is exothermic, then solubility will decrease with increasing temperature
	\item A solution can be heated to dissolve more solute, then cooled to form a supersaturated solution
	\item Solute can be spontaneously precipitated from a supersaturated solution -- Demo with sodium acetate
\end{itemize}

\section{Pressure Dependence of Solubility}
\begin{itemize}
	\item Pressure will affect the solubility of gases in liquids
	\item The solubility is proportional to the gas pressure, according to Henry's Law
\end{itemize}

\section{Units of Concentration}
\begin{itemize}
	\item There are several different units used by chemists to describe concentration
	\item Percent concentrations:
	      \begin{itemize}
		      \item mass/mass percent -- mass of solute / mass of solution
		      \item volume/volume percent -- volume of solute
		      \item mass/volume percent -- mass of solute volume of solution
	      \end{itemize}
\end{itemize}

\section*{Homework 9.1:} % TODO: Check these and all following questions
\begin{itemize}
  \item 9.38 -- Water as a solvent
  \item 9.42 -- Henry's law
  \item 9.44 -- Saturation and solubility
  \item 9.50 -- Preparing a solution of \ch{NaCl}
\end{itemize}

\section{Dilution}
\begin{itemize}
	\item Dilution is when additional solvent is added to reduce the concentration of all solutes
	\item The number of moles of solvent doesn't change, so we can set the number of moles after dilution equal to the number of moles before the dilution
	\item $C_1V_1=C_2V_2$ in general, or $M_1V_1=M_2V_2$ for molar concentrations
	\item $\dfrac{V_1}{V_2}$ is the dilution factor
\end{itemize}

\section{Ions in Solution: Electrolytes}
\begin{itemize}
	\item Substances which produce ions when dissolved in solutions are called electrolytes
	\item Electrolytes conduct electricity
	\item Substances fall into three categories:
	      \begin{itemize}
		      \item Strong electrolytes completely ionize (produce one or more ions for each formula unit)
		      \item Weak electrolytes only partially ionize (often only a fraction of a percent)
		      \item Non-electrolytes produce no ions
	      \end{itemize}
\end{itemize}

\section*{Electrolytes in Biology}
\begin{itemize}
	\item In our biochemistry, there are often many different ions in solution
	\item To talk about the concentration of complex solutions, we use two new units:
	      \begin{itemize}
		      \item Equivalents (Eq): The number of ions that carry 1 mole of charge
		      \item Gram Equivalents (g-Eq): $\dfrac{\mathrm{Molar Mass of Ion (g)}}{\mathrm{Charge on Ion}}$
	      \end{itemize}
	\item More realistic concentrations are milliequivalents
\end{itemize}

\section{Properties of Solutions}
\begin{itemize}
	\item Some properties of solutions depend on the concentration but not the identity of the solute
	\item These properties are called colligative properties
	\item Vapor-Pressure Lowering -- Adding a solute will lower the solvent vapor pressure
	\item Boiling Point Elevation -- $\Delta T_b = \kappa_b C_{molal}$
	\item Freezing Point Depression -- $\Delta T_f = \kappa_f C_{molal}$
\end{itemize}

\section{Osmosis and Osmotic Pressure}
\begin{itemize}
	\item Osmotic pressure is another important colligative property
	\item Osmosis -- Pressure exerted across a semipermeable membrane $\pi = \dfrac{nRT}{V}$
	\item Isotonic solutions match the salinity of blood cells
	\item hypertonic solutions are more salty and will shrivel blood cells
	\item hypotonic solutions are less salty and will swell (or lyse) blood cells
\end{itemize}

\section{Dialysis} % TODO: This section is new

\section*{Homework 9.2:}
\begin{itemize}
  \item 9.68 -- Dilution
  \item 9.74 -- milliequivalents
  \item 9.82 -- Freezing point depression
  \item 9.86 -- Osmolarity
\end{itemize}

\chapter{Acids and Bases} %TODO: It looks like a lot of restructuring in this chapter. Do a full run-through
\section{Acids and Bases in Aqueous Solution}
\begin{itemize}
	\item Acid and base activity is mostly defined in the context of an aqueous solution
	\item An acid will react with water to produce hydronium ions
	\item A base will react with water to produce hydroxide ions
\end{itemize}

\section{Some Common Acids and Bases}
\begin{itemize}
	\item Acids and bases are common in products around us. Page 292 lists a few
\end{itemize}

\section{The Br\o nsted-Lowry Definition of Acids and Bases}
\begin{itemize}
	\item There are several different definitions used for acids and bases, we will be using the Br\o nsted-Lowry definition:
	      \begin{itemize}
		      \item Remember that a \ch{H^+} ion is simply a proton
		      \item An acid is a proton donor -- \ch{HCl} reaction with water
		      \item A base is a proton acceptor -- \ch{NH3} reaction with water
	      \end{itemize}
	\item Under this definition, water acts as a base when reacting with an acid, and acts as an acid when reacting with a base -- More on this later
	\item Multiprotic acids can donate more than one hydrogen
	\item Every acid has a conjugate base, and every base has a conjugate acid -- They make up a conjugate pair
\end{itemize}

\section{Acid and Base Strength}
\begin{itemize}
	\item Strong acids and bases will dissociate into \ch{H^+} or \ch{OH^-} completely (and are strong electrolytes)
	\item Weak acids and bases dissociate to different extents, less than 100\%
	\item The relative strengths of a conjugate pair are inverse to each other (i.e. The stronger and acid, the weaker will be its conjugate base, and vice-versa)
	\item Table 10.1 shows conjugate pairs over a range of acid and base strengths
	\item Acid-base reactions are equilibrium reactions, and the relative strengths determine whether the reaction is reactant or product favored
\end{itemize}

\section{Acid Dissociation Constants}
\begin{itemize}
	\item The strength of an acid is quantified by the acid dissociation constant $K_a$
	\item This is really just th equilibrium constant for the reaction of acid with water
	\item $K_a=\dfrac{[\ch{H3O^+}][\ch{A^-}]}{[\ch{HA}]}$
	\item Multiprotic acids will have a seperate constant for each stage of dissociation
	\item Table 10.2 Gives the $K_a$ values for several common acids
\end{itemize}

\section{Water as Both an Acid and a Base}
\begin{itemize}
	\item Water acts as a base when reacting with an acid, and acts as an acid when reacting with a base
	\item Substances like this are called \emph{amphoteric}
	\item Water can also react with itself, acting as both an acid and a base to produce both hydronium and hydroxide
	\item The water dissociation constant $K_w$ is very important to acid-base chemistry and has the value $[\ch{H3O^+}][\ch{OH^-}]=1.0\times10^{-14}$
	\item In a neutral solution, $[\ch{H3O^+}] = [\ch{OH^-}] = 1.0\times10^{-7}$
	\item As $[\ch{H3O^+}]$ goes up, $[\ch{OH^-}]$ goes down and vice-versa to obey the $K_w$ expression
	\item Homework 10.1:
	      \begin{itemize}
		      \item 10.44 -- Identify strong acids
		      \item 10.46 -- Identify acids and bases
		      \item 10.54 -- Equilibrium expression $K_a$
		      \item 10.56 -- $K_w$
	      \end{itemize}
\end{itemize}

\section{Measuring Acidity: pH}
\begin{itemize}
	\item pH is a way to quantify the acidity of a solution
	\item $pH=-\log_{10}[\ch{H3O^+}]$
	\item Taking the log squeezes the very wide range of possible hydronium concentrations into a reasonably small range.
	\item $pH=7$ is neutral, $pH<7$ is acidic, and $pH>7$ is basic
	\item We can also define a pOH: $pOH=-\log_{10}[\ch{OH^-}]$
	\item With these definitions we can convert $K_w=[\ch{H3O^+}][\ch{OH^-}]$ into $14=pH+pOH$
\end{itemize}

\section{Working with pH}
\begin{itemize}
	\item We find pH by taking the negative log of the hydronium concentration
	\item We can find the hydronium concentration by taking ten to the power of -pH
	\item Draw my box of conversions
\end{itemize}

\section{Laboratory Determination of Acidity}
\begin{itemize}
	\item One simple way to measure acidity is with a color indicator (Figure 10.3)
	\item This can take the form of a drop to add to a solution, or a piece of litmus paper to drop the solution onto
	\item Indicators are chemicals that change color based on the acidity of their environment
	\item More accurate measurements can be made electronically using a pH meter
\end{itemize}

\section{Buffer Solutions}
\begin{itemize}
	\item Pure water will rapidly change pH if any acid or base is added
	\item Buffer solutions can withstand changes in pH, even if strong acids or bases are added
	\item Figure 10.5 compares pH changes in pure water vs a buffer solution
	\item A buffer solution contains a conjugate acid/base pair, which reacts in place of the water to resist pH changes
	\item To make an ideal buffer, pick an acid/base pair whose $pK_a$ equals the desired pH you want to buffer at
	\item The Henderson-Hasselbalch equation helps to solve for the pH of a buffer solution that doesn't have exactly equal amounts of the conjugate acid and base
	\item $pH=pK_a + \log_{10}\left(\dfrac{[\ch{A^-}]}{\ch{[HA]}}\right)$
	\item Our own bodies are regulated by buffer systems (carbonic acid and dihydrogen phosphate, among others)
\end{itemize}

\section{Acid and Base Equivalents}
\begin{itemize}
	\item Just like with ion equivalents, it is often convenient to talk about acids and bases in terms of how many moles of \ch{H3O^+} of \ch{OH^-} they will produce
	\item $1eq = \dfrac{1mol}{\#~of~\ch{H3O^+}~or~\ch{OH^-}~ions~produced}$
	\item $1g-eq = \dfrac{Molar Mass}{\#~of~\ch{H3O^+}~or~\ch{OH^-}~ions~produced}$
	\item Since acids and bases will partition between their conjugates, we can't properly talk about Molarity without doing further calculations
	\item What we really want is the number of equivalents in a solution. This is called the \emph{Normal Concentration} (N)
	\item $N=\dfrac{eq}{L}$
\end{itemize}

\section{Some Common Acid-Base Reactions}

\section{Titration}
\begin{itemize}
	\item Titration is a technique used to determine the concentration of acids and bases
	\item To titrate an acid, a base of known concentration is added dropwise until the the acid is neutralized
	\item To titrate a base, an acid of known concentration is added dropwise until the the base is neutralized
	\item At the equivalence point, the number of moles of acid and base should be equal
	\item The equivalence point can be identified by a color indicator, since the pH swings rapidly once the neutralization has occurred
	\item $N_1V_1=N_2V_2$ -- This looks just like the dilution equation because it is based on the same assumption: That the number of moles is the same
\end{itemize}

\section{Acidity and Basicity of Salt Solutions}
\begin{itemize}
	\item When a salt dissolves, the cation and anion dissociate and are free to react with water
	\item The cation may be an acid, and the anion may be a base or an acid, so we need to consider them both
	\item Remember that a salt is the product of an acid-base reaction, so we can classify salts by the type of reactions which will produce them:
	      \begin{itemize}
		      \item Strong acid + weak base = acidic solution
		      \item Weak acid + strong base = basic solution
		      \item Strong acid + strong base = neutral solution
		      \item Weak acid + weak base = ? (it depends on the specific acid/base strengths)
	      \end{itemize}
	\item Homework 10.2:
	      \begin{itemize}
		      \item 10.68 -- Rough calculations of pH
		      \item 10.70 -- pH to $\left[\ch{H3O^+}\right]$ and $\left[\ch{OH^-}\right]$
		      \item 10.78 -- Find pH of a buffer
		      \item 10.90 -- Titration
	      \end{itemize}
\end{itemize}

\chapter{Nuclear Chemistry}
\section{Nuclear Reactions}
\begin{itemize}
	\item Atomic symbols keep track of all subatomic particles: \ch{^{12}_6C}, \ch{^{13}_6C}, \ch{^{12}_6C^+}, etc.
	\item We can describe changes in nuclear reactions using these symbols:

	      \ch{^{14}_6C -> ^{14}_7N + ^0_{-1}e^{-}}
	\item In the reaction above, \ch{^0_{-1}e^{-}} (simply an electron) does not really have an atomic number, but the \ch{_{-1}} is useful for balancing reactions
\end{itemize}

\section{The Discovery and Nature of Radioactivity}
\begin{itemize}
	\item Radioactivity was first observed in 1896 when photosensitive paper stored under a Uranium sample was exposed despite being kept in the dark
	\item Radiation is categorized as $\alpha$ particles, $\beta$ particles, or $\gamma$ radiation
	\item Figure 11.1 shows how different types of radiation respond to an electric field
	\item Table 11.1 shows the physical properties of different types of radiation
\end{itemize}

\section{Stable and Unstable Isotopes}
\begin{itemize}
	\item Radioactivity comes from unstable nuclei
	\item Some unstable nuclei are naturally occurring (\ch{U}), while others \ch{(Pu)} are generated artificially
	\item Figure 11.2 shows the ``belt of stability''
\end{itemize}

\section{Nuclear Decay}
\begin{itemize}
	\item Alpha Emission -- Figure 11.3

	      \ch{^{238}_{92}U -> ^4_2He + ^{234}_{90}Th}

	      \ch{^{208}_{84}Po -> ?}
	\item Beta Emission -- ~~ \ch{^1_0n -> ^1_1p^+ + ^0_{-1}e^{-}}

	      \ch{^{131}_{53}I -> ^{131}_{54}Xe + ^0_{-1}e^-}

	      \ch{^{55}_{24}Cr -> ?}
	\item Gamma Emission -- Often present with other decay (Exception is Hafnium battery)

	      \ch{^{60}_{27}Co -> ^{60}_{28}Ni + ^0_{-1}e^- + ^0_0$\gamma$}
	\item Positron Emission -- ~~ \ch{^1_1p^+ -> ^1_0n + ^0_1e^+}

	      \ch{^{40}_{19}K -> ^{40}_{18}Ar + ^0_1e^+}
	\item Electron Capture -- ~~ \ch{^1_1p^+ + ^0_{-1}e^- -> ^1_0n}

	      \ch{^{197}_{80}Hg + ^0_{-1}e^- -> ^{197}_{79}Au}
\end{itemize}

\section{Radioactive Half-Life}
\begin{itemize}
	\item The half-life is the time required for half of a radioactive sample to decay
	\item Figure 11.4 shows how the absolute decay rate decreases, but the half-life is constant
	\item $Fraction~Remaining = \left(0.5\right)^{n}$
	\item Table 11.3 shows the half-lives and uses for common radioisotopes
\end{itemize}

\section*{Radioactive Decay Series}
\begin{itemize}
	\item Radioisotopes often decay into other radioisotopes, creating a cascade of different reactions
	\item Figure 11.5 shows the Uranium radioactive decay series
\end{itemize}

\section{Ionizing Radiation}
\begin{itemize}
	\item Radioactive products are most damaging when they eject electrons off of other materials
	\item Table 11.4 shows the penetrating distance of different types of ionizing radiation
\end{itemize}

\section{Detecting and Measuring Radiation}

\section{Artificial Transmutation}

\section{Nuclear Fission and Nuclear Fusion}
\begin{itemize}
	\item Fission and fusion both release tremendous amounts of energy, which manifest as lost mass according to $E=mc^2$
	\item Fission -- Figure 11.7

	      \ch{^1_0n + ^{235}_{92} U -> ^{142}_{56}Ba + ^{91}_{36}Kr + 3 ^1_0n}
	\item Note that the production of 3 neutrons can lead to a chain reaction
	\item Fusion -- Combining smaller nuclei to form larger ones

	      \ch{^1_1H + ^2_1H -> ^3_2He}

	      \ch{^3_2He + ^3_2He -> ^4_2He + 2 ^1_1H}

	      \ch{^3_2He + ^1_1H -> ^4_2He + ^0_1e^+}
	\item Homework 11.1:
	      \begin{itemize}
		      \item 11.40 -- $\alpha$ and $\beta$ decay
		      \item 11.44 -- Predicting products of $\beta$ decay
		      \item 11.48 -- Balancing fission reactions
		      \item 11.56 -- Half-life calculations
	      \end{itemize}
\end{itemize}
\end{document}
