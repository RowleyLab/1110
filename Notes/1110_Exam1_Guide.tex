\documentclass[12pt, letterpaper]{memoir}
\usepackage{ExamStyle}

\begin{document}
	\mainmatter
	
	\begin{center}
		{\Huge CHEM 1110}
		
		{\LARGE Midterm Exam 1 Study Guide (Ch. 1-2)}
	\end{center}
	
	This study guide is meant to provide only the barest direction as you study. Try to find practice problems from the textbook (both in the chapter text and in the end-of-chapter questions) rather than just relying on this guide. Note that most tables and equations will not be provided here, or on the exam. You can find them in your textbook now, but should memorize them in preparation for the exam.
	

	\subsection*{Chapter 1 -- Matter and Measurements}
	\begin{itemize}
		\item Physical and chemical properties
		\item Physical and chemical changes
		\item States of matter and state changes
		\item Elements, compounds, homogeneous mixtures, and heterogeneous mixtures
		\item Chemical reactions (reactants and products)
		\item Chemical formulas
		\item Finding derived units from measurements (e.g. density from mass and volume)
		\item Converting between units (be careful for squared and cubed units)
		\item Use of metric prefixes and scientific notation to describe very large or very small numbers
		\item Precision vs. accuracy
		\item Significant figures in a number
		\item Propogating significant figures in mathematical operations ($+-$ rule and $\times \div$ rule)
		\item Solving problems through dimensional analysis
		\item Temperature changes with heat transfer; Heat capacities
	\end{itemize}
	\subsection*{Chapter 2 -- Atoms and the Periodic Table}
	\begin{itemize}
		\item Atomic theory
		\item Atomic mass unit
		\item Subatomic particles and their interactions		
		\item Modern view of an atom -- nucleus with electron cloud
		\item Neutrons, electrons, and protons
		\begin{itemize}
			\item Neutrons and protons make up most of the mass
			\item Electrons make up most of the volume
			\item Charges of electrons and protons
			\item Protons define the element
		\end{itemize}
		\item Writing atomic symbols from numbers of electrons, neutrons, and protons and vice-versa
		\item Calculating atomic weights from isotope mass and percent abundance
		\item The periodic table
		\begin{itemize}
			\item Names of certain families (alkali, alkaline earth, halogens, and noble gases)
			\item Metals vs. non-metal vs. metalloids
			\item Transition metals vs. main group elements
		\end{itemize}
		\item Quantized electronic energy levels - wavelike nature of electrons
		\item Shells and subshells - s, p, d, and f subshells
		\item Orbitals and spin - 2 electrons per orbital
		\item Electron configurations
		\begin{itemize}
			\item Energy level diagrams
			\item Noble-gas condensed configurations
		\end{itemize}
		\item Electronic structure and the structure of the periodic table
		\item Core vs valence electrons
		\item Electron dot symbols
	\end{itemize}	
\end{document}