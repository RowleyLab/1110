\documentclass[12pt, letterpaper]{memoir}
\usepackage{ExamStyle}

\begin{document}
	\mainmatter
	
	\begin{center}
		{\Huge CHEM 1110}
		{\LARGE-- Fall 2021
		
		Midterm Exam 3 Study Guide (Ch. 6-8)}
	\end{center}
	
	This study guide is meant to provide only the barest direction as you study. Try to find practice problems from the textbook (both in the chapter text and in the end-of-chapter questions) rather than just relying on this guide. Note that most tables and equations will not be provided here or on the exam. You can find them in your textbook. 

	\subsection*{Chapter 6 -- Chemical Reactions: Mole and Mass Relationships}
	\begin{itemize}
		\item Calculating molecular weights and formula weights
		\item Calculating moles from mass, and mass from moles
		\item Stoichiometric ratios between reactants and products
		\item Calculating theoretical yield
		\item Limiting reactant problems
		\begin{itemize}
			\item Which is the limiting reactant?
			\item How much excess reactant is left over?
			\item What is the percent yield?
		\end{itemize}
	\end{itemize}

	\subsection*{Chapter 7 -- Chemical Reactions: Energy, Rates, and Equilibrium}
	\begin{itemize}
		\item Energy is released by forming bonds and energy is required to break bonds
		\item Finding enthalpies of reaction from bond enthalpies
		
		$\Delta H_{rxn} \approx \displaystyle\sum Bonds~Broken - \displaystyle\sum Bonds~Formed$
		
		\item Find the enthalpy of reaction for the following reaction:
		
		\ch{H2(g) + Cl2(g) -> 2 HCl}
		\item Answer: $\Delta H_{rxn} \approx -43~\dfrac{kcal}{mol}$
		\item Question: If $5.00~g$ of \ch{H2} react with excess \ch{Cl2} to produce \ch{HCl}, how much heat is released?
		\item Answer: $107~kcal$
		\item Exothermic vs Endothermic reactions
		\item Free energy and spontanaety (negative $\Delta G_{rxn} is spontaneous$)
		
		$\Delta G_{rxn} = \Delta H_{rxn} - T \Delta S_{rxn}$
		\item Suppose a reaction has $\Delta H_{rxn}>0$ and $\Delta S_{rxn}>0$. Will this reaction be spontaneous at high, low, all, or no temperatures?
		\item Answer: High temperature
		\item Rates of chemical reactions are controlled by three factors:
		\begin{itemize}
			\item The frequency of collisions - Concentration of reactants
			\item The energy of collisions - Must be greater than activation energy. Temperature dictates average energy
			\item Reaction pathway - A catalyst could speed a reaction
		\end{itemize}
		\item True/False Increasing the temperature will decrease the reaction rate for exothermic reactions.
		\item Answer: False. Increasing the temperature always increases the reaction rate.
		\item Reaction coordinate diagrams
		\item Some reactions are \emph{reversible}. They reach an equilibrium rather than going all the way to products
		\item The equilibrium conditions are governed by the equilibrium constant: $K_{eq}=\dfrac{[Products]^m}{[Rreactants]^n}$
		\item Give the equilibrium constant for the following reaction:
		
		\ch{PO4^{3-}(aq) + 2 H2O(l) <-> H2PO4^{-}(aq) + 2 OH^{-}(aq)}
		\item Answer: $K_{eq}=\dfrac{[H_2PO_4^{-}][OH]^2}{[PO4^{3-}]}$ (note that the water is absent because it is (l))
		\item Le Ch\^atelier's Principle:
		\begin{itemize}
			\item Shifting in response to adding or removing reactants or products
			\item Shifting in response to changes in pressure
			\item Shifting in response to changes in temperature
		\end{itemize}
	\end{itemize}
	
	\subsection*{Chapter 8 -- Gases, Liquids, and Solids}
	\begin{itemize}
		\item $\Delta H$, $\Delta S$, and $\Delta G$ for all phase changes
		\item Intermolecular forces
		\begin{itemize}
			\item London Dispersion Forces
			\item Dipole-dipole Forces
			\item Hydrogen Bonds
		\end{itemize}
		\item Question: Is methanol (\ch{CH3OH}) capable of hydrogen bonding?
		\item Answer: Yes, it has both a H-bond donor and acceptor, so it can hydrogen bond
		\item Know the assumptions that make up the kinetic molecular theory of gases
		\item Pressure an the many different units for expressing it
		\item Gas laws: $\dfrac{P_1V_1}{T_1}=\dfrac{P_2V_2}{T_2}$
		\item Question: A balloon has $V=3.21~L$ at $P=0.82~atm$ and $T=298~K$. What temperature would give the balloon a volume of $V=2.95~L$ at $P=0.67~atm$?
		\item Answer: $T=224~K$
		\item Ideal Gas Law: $PV=nRT$
		\item Dalton's law of partial pressures
		\item Vapor pressures, boiling points, and Henry's law
		\item Viscosity and surface tension
		\item Different types of crystalline solids and amorphous solids
		\item Heats for phase changes: $q=m\Delta H$
		\item Question: How much heat is absorbed when $12.3~g$ of dry ice sublimate in your home-made root beer? ($\Delta H_{sub} = 571\dfrac{J}{g}$)
		\item Answer: $7.02~kJ$
	\end{itemize}
\end{document}