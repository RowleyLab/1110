\documentclass[12pt, letterpaper]{memoir}
\usepackage{ExamStyle}

\begin{document}
	\mainmatter
	
	\begin{center}
		{\Huge CHEM 1110}
		
		{\LARGE Midterm Exam 3 Study Guide (Ch. 5-7)}
	\end{center}
	
		This study guide is meant to provide only the barest direction as you study. Try to find practice problems from the textbook (both in the chapter text and in the end-of-chapter questions) rather than just relying on this guide. Note that most tables and equations will not be provided here, or on the exam. You can find them in your textbook now, but should memorize them in preparation for the exam.

	\subsection*{Chapter 5 --  Classification and Balancing of Chemical Reactions}
	\begin{itemize}
		\item How to balance chemical reactions
		\begin{itemize}
			\item Start with elements that appear in \emph{only} one place on each side of the equation
			\item Often it is best to balance oxygen last
			\item double everything if necessary to get rid of fractional coefficients
		\end{itemize}
		\item Classes of chemical reactions
		\begin{itemize}
			\item Precipitation reactions
			\item Acid-base reactions
			\item Oxidation-reduction reactions
		\end{itemize}
		\item Precipitation reactions
		\begin{itemize}
			\item Solubility rules in Table 5.1
			\item Reactants will switch binding partners (cations and anions)
			\item Check the solubility of the new products
			\item Net ionic equations will eliminate the spectator ions
		\end{itemize}
		\item Acid/base neutralization reactions
		\begin{itemize}
			\item Acids produce \ch{H3O^{+}} in water
			\item Bases produce \ch{OH^{-}} in water
			\item Acids and bases react together to produce neutral solutions with salt, and sometimes water
		\end{itemize}
		\item Reduction-oxidation reactions
		\begin{itemize}
			\item Redox reactions involve the transfer of electrons
			\item Electrons are tracked with oxidation numbers
			\begin{itemize}
				\item Elemental forms have oxidation states of 0
				\item H usually takes +1 and O usually takes -2
				\item The sum of all oxidation numbers should equal the overall charge
			\end{itemize}
			\item OIL-RIG -- Oxidation is losing electrons, reduction is gaining electrons
			\item See how oxidation states change to identify which elements were oxidized and which were reduced
			\item Oxidizing agents and reducing agents are identified by their effect on their reaction partner
		\end{itemize}
	\end{itemize}

	\subsection*{Chapter 6 -- Chemical Reactions: Mole and Mass Relationships}
	\begin{itemize}
		\item Calculating molecular weights and formula weights
		\item Calculating moles from mass, and mass from moles
		\item Stoichiometric ratios between reactants and products
		\item Calculating theoretical yield
		\item Limiting reactant problems
		\begin{itemize}
			\item Which is the limiting reactant?
			\item How much excess reactant is left over?
			\item What is the percent yield?
		\end{itemize}
	\end{itemize}

	\subsection*{Chapter 7 -- Chemical Reactions: Energy, Rates, and Equilibrium}
	\begin{itemize}
		\item Energy is released by forming bonds and energy is required to break bonds
		\item Finding enthalpies of reaction from bond enthalpies
		
		$\Delta H_{rxn} \approx \displaystyle\sum Bonds~Broken - \displaystyle\sum Bonds~Formed$
		
		\item Find the enthalpy of reaction for the following reaction:
		
		\ch{H2(g) + Cl2(g) -> 2 HCl}
		\item Answer: $\Delta H_{rxn} \approx -43~\dfrac{kcal}{mol}$
		\item Question: If $5.00~g$ of \ch{H2} react with excess \ch{Cl2} to produce \ch{HCl}, how much heat is released?
		\item Answer: $107~kcal$
		\item Exothermic vs Endothermic reactions
		\item Free energy and spontanaety (negative $\Delta G_{rxn} is spontaneous$)
		
		$\Delta G_{rxn} = \Delta H_{rxn} - T \Delta S_{rxn}$
		\item Suppose a reaction has $\Delta H_{rxn}>0$ and $\Delta S_{rxn}>0$. Will this reaction be spontaneous at high, low, all, or no temperatures?
		\item Answer: High temperature
		\item Rates of chemical reactions are controlled by three factors:
		\begin{itemize}
			\item The frequency of collisions - Concentration of reactants
			\item The energy of collisions - Must be greater than activation energy. Temperature dictates average energy
			\item Reaction pathway - A catalyst could speed a reaction
		\end{itemize}
		\item True/False Increasing the temperature will decrease the reaction rate for exothermic reactions.
		\item Answer: False. Increasing the temperature always increases the reaction rate.
		\item Reaction coordinate diagrams
		\item Some reactions are \emph{reversible}. They reach an equilibrium rather than going all the way to products
		\item The equilibrium conditions are governed by the equilibrium constant: $K_{eq}=\dfrac{[Products]^m}{[Rreactants]^n}$
		\item Give the equilibrium constant for the following reaction:
		
		\ch{PO4^{3-}(aq) + 2 H2O(l) <-> H2PO4^{-}(aq) + 2 OH^{-}(aq)}
		\item Answer: $K_{eq}=\dfrac{[H_2PO_4^{-}][OH]^2}{[PO4^{3-}]}$ (note that the water is absent because it is (l))
		\item Le Ch\^atelier's Principle:
		\begin{itemize}
			\item Shifting in response to adding or removing reactants or products
			\item Shifting in response to changes in pressure
			\item Shifting in response to changes in temperature
		\end{itemize}
	\end{itemize}
	
	
\end{document}