\documentclass[12pt, letterpaper]{memoir}
\usepackage{ExamStyle}

\begin{document}
	\mainmatter
	
	\begin{center}
		{\Huge CHEM 1110}
		{\LARGE-- Fall 2021
		
		Midterm Exam 4 Study Guide (Ch. 9-11)}
	\end{center}
	
	This study guide is meant to provide only the barest direction as you study. Try to find practice problems from the textbook (both in the chapter text and in the end-of-chapter questions) rather than just relying on this guide. Note that most tables and equations will not be provided here or on the exam. You can find them in your textbook. 
	
	\subsection*{Chapter 9 --  Solutions}
	\begin{itemize}
		\item Solutions vs colloids
		\item Factors leading to solubility
		\item Recognize hydrates and using their molar mass properly
		\item Saturation and super-saturation
		\item Temperature dependence of solubility
		\item Henry's law and pressure dependence of solubility
		\item Different units of concentration
		\begin{itemize}
			\item mass/mass percent
			\item volume/volume percent
			\item Molarity
			\item Molality
		\end{itemize}
		\item Question: $2.6~g$ of \ch{NaCl} are dissolved in $63.4~g$ of water. What is the concentration in mass/mass percent, and in molality?
		\item Answer: $3.9\%$ and $0.70~molal$ in \ch{NaCl}, but $1.4~molal$ overall since \ch{NaCl} dissociates into two different parts
		\item Dilution
		\item Question: How many $ml$ of a $12.3~M$ stock solution should you use to make $200.0~ml$ of a $0.500~M$ solution?
		\item Answer: $8.13~ml$
		\item Strong vs weak electrolytes
		\item Equivalents and gram-equivalents
		\item Question: How many equivalents of positive charge are in $15.0~g$ of \ch{Ca(NO3)2}?
		\item Answer: $0.183~eq$
		\item Question: How much is a gram-equivalent for the \ch{Ca^{2+}} ion?
		\item Answer: $20.0~grams$
		\item Boiling point elevation and freezing point depression: $\Delta T=\kappa C_{molal}$
		\item Question: You place $3.5~g$ of \ch{MgCl2} in a pot with $62~g$ of water. What are the new boiling point and freezing point for the water? (for water, $\kappa_b = 0.512\dfrac{^\circ C}{m}$ and $\kappa_f = -1.86\dfrac{^\circ C}{m}$)
		\item Answer: $T_b = 100.91 ^\circ C$ and $T_f = -3.3 ^\circ C$
		\item Osmotic pressure: $\pi = \dfrac{nRT}{V}$
	\end{itemize}

	\subsection*{Chapter 10 --  Acids and Bases}
	\begin{itemize}
		\item Br\o nsted-Lowry Definition of Acids and Bases
		\item Acid/Base reactions with water
		\item Identify acid, base, conjugate acid, and conjugate base in acid/base reactions
		\item Weak vs. strong acids and bases (know the strong acids)
		\item Acid/base strength, and relationship between strength of conjugate pairs
		\item Acid dissociation constant $\left(K_a=\dfrac{[\ch{H3O^+}][\ch{A^-}]}{[\ch{HA}]}\right)$
		\item Using $K_w$ to find $[\ch{H3O^+}]$ or $[\ch{OH^-}]$
		\item Finding $pH$, $pOH$, and $pK_a$
		\item Use of color indicators
		\item Purpose and composition of a buffer solution
		\item Finding pH for a buffer solution (Henderson-Hasselbalch equation)
		\item Finding concentrations by titration
		\item Predicting acid/base properties of ionic compounds
	\end{itemize}

	\subsection*{Chapter 11 --  Nuclear Chemistry}
	\begin{itemize}
		\item Differences between $\alpha$, $\beta$, and $\gamma$ radiation
		\item Balancing nuclear equations
		\begin{itemize}
			\item $\alpha$ emission
			\item $\beta$ emission
			\item Positron emission
			\item Electron capture
			\item Fission
			\item Fusion
		\end{itemize}
		\item Half-life and finding a sample's remaining fraction
	\end{itemize}
\end{document}