\documentclass[12pt, letterpaper]{memoir}
\usepackage{ExamStyle}

\begin{document}
	\mainmatter
	
	\begin{center}
		{\Huge CHEM 1110}
		
		{\LARGE Midterm Exam 2 Study Guide (Ch. 3-4)}
	\end{center}
	
		This study guide is meant to provide only the barest direction as you study. Try to find practice problems from the textbook (both in the chapter text and in the end-of-chapter questions) rather than just relying on this guide. Note that most tables and equations will not be provided here, or on the exam. You can find them in your textbook now, but should memorize them in preparation for the exam.

	\subsection*{Chapter 3 -- Ionic Compounds}
	\begin{itemize}
		\item Recognizing ionic compounds
		\begin{itemize}
			\item Metals bonded to non-metals
			\item Polyatomic ions often don't include a metal
		\end{itemize}
		\item Cations and anions
		\item Predicting ion charge based on position in the periodic table
		\begin{itemize}
			\item Metals produce cations, non-metals produce anions
			\item Elements will lose or gain enough electrons to satisfy the \emph{octet} rule
			\item Transition metals might take different charges in different compounds
		\end{itemize}
		\item Ionic bonds create a crystal lattice (no discrete molecular unit)
		\item Ionic compounds are rigid, but brittle, and have high melting points
		\item Naming ions
		\begin{itemize}
			\item Cations are just the metal name
			\item For transition metals, include the charge in roman numerals, like this: Iron(III) for \ch{Fe^{3+}}
			\item Anions end in ``-ide''
			\item Polyatomic ions have their own names, but often end in ``-ate'' or ``-ite'' (table 3.3)
		\end{itemize}
		\item Ions combine in definite ratios to form ionic compounds
		\begin{itemize}
			\item The total positive and negative charges need to balance
			\item Find the least common multiple to determine the right numbers of each ion
		\end{itemize}
		\item Naming ionic compounds
		\begin{itemize}
			\item Just combine the two ions names -- Cation first, then anion
			\item To determine the charge on a transition metal, look at the total amount of negative charge that it must balance
		\end{itemize}
	\end{itemize}
	
	\subsection*{Chapter 4 -- Molecular Compounds}
	\begin{itemize}
		\item Covalent bonds involve \emph{sharing} electrons
		\item Compounds with covalent bonds are called \emph{molecules}
		\item Forces involved in a covalent bond:
		\begin{itemize}
			\item The positively charged nuclei repulse each other when they are too close
			\item There are no interactions at all when they are too far
			\item At the right bond distance, the electrons are attracted to \emph{both} nuclei, forming the bond
		\end{itemize}
		\item Electrons are shared when orbitals from both bonding partners overlap
		\item Predicting the number of bonds based on position in the periodic table
		\begin{itemize}
			\item Elements will form enough bonds to satisfy the octet rule
			\item Each bond brings one additional electron
			\item This trend is not a hard rule, and some compounds will form many more bonds than expected
		\end{itemize}
		\item Multiple bonds
		\begin{itemize}
			\item Sometimes molecules need to bond multiple times to the same partner in order to satisfy the octet rule
			\item Consider \ch{O2} (double bond) and \ch{N2} (triple bond)
			\item Each double bond shares 4 electrons, and each triple bond shares 6
		\end{itemize}
		\item For coordinate covalent bonds, both electrons come from the same atom
		\item Molecular compounds have widely ranging properties, such as melting points and reactivities
		\item Ways of representing molecular compounds:
		\begin{itemize}
			\item Chemical formula
			\item Condensed structural formula
			\item Lewis structure
		\end{itemize}
		\item Drawing Lewis Structures
		\begin{itemize}
			\item 5 steps to drawing a Lewis structure:
			\begin{enumerate}
				\item Count up the total number of valence electrons (taking into account any net charge)
				\item Draw single bonds from the central atom to all peripheral atoms
				\item Fill the octets of the outer atoms with lone-pairs (H doesn't need any)
				\item Place any remaining electrons onto the central atom
				\item Satisfy the octet rule for the central atom, if needed, by converting outer lone-pairs into multiple-bonds with the central atom
			\end{enumerate}
			\item Finding the geometry from the Lewis Structure
			\begin{enumerate}
				\item Draw a valid Lewis structure
				\item Count the number of electron domains
				\item Count the number of bonds vs. lone pairs
				\item Consult Table 4.2 (or your memory)
			\end{enumerate}
		\end{itemize}
		\item Electronegativity, and how it varies across the periodic table
		\item Polar vs non-polar bonds
		\item polar vs non-polar molecules
		\begin{itemize}
			\item lone pairs disrupt symmetry and lead to polar molecules (\ch{NH3})
			\item Bonds to different atom types disrupt symmetry and lead to polar molecules (\ch{CH2O})
		\end{itemize}
		\item Naming binary molecular compounds
		\begin{itemize}
			\item The first element named is the least electronegative
			\item The second element named ends with ``-ide''
			\item Indicate how many of each element with the prefixes in table 4.3
			\item Leave off ``mono-'' for the first element only
		\end{itemize}
	\end{itemize}

	
\end{document}