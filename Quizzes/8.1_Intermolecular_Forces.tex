\documentclass[11pt, letterpaper]{memoir}
\usepackage{HomeworkStyle}
\geometry{margin=0.75in}



\begin{document}

	\begin{center}
		{\large Quiz 8.1 --	Intermolecular Forces and Gases}
	\end{center}
	{\large Name: \rule[-1mm]{4in}{.1pt} 

\subsection*{Question 1}
State the strongest intermolecular force exhibited by each molecule:

{\large \ch{CH3CO2H} \hspace{3em} \ch{O3} \hspace{3em} \ch{NO2} \hspace{3em} \ch{C3H8} \hspace{3em} \ch{CH3OH} \hspace{3em} \ch{CH2F2} \hspace{3em} \ch{N2}}

\vspace{2em}
\subsection*{Question 2}
Which substance in each pair would have the highest melting point and boiling point:

{\large \ch{Kr~/~Ne} \hspace{3em} \ch{NO2~/~N2O4} \hspace{3em} \ch{O2~/~N2} \hspace{3em} \ch{CH3CH2CH2CH3~/~CH(CH3)3} \hspace{3em} \ch{C6H14~/~C8H18}}

\vspace{2em}
\subsection*{Question 3}
Draw and describe the two features that a substance must have to be capable of forming \ch{H}-bonds

\vspace{4em}
\subsection*{Question 4}
List the five postulates of the kinetic molecular theory:


\vspace{5em}
\subsection*{Question 5}
Complete the following pressures into $atm$

\begin{itemize}
	\item $224~mm\ch{Hg}$
	
	\vspace{1em}
	\item $65~millitorr$
\end{itemize}

\vspace{1em}
\subsection*{Question 6}
Convert $1.25~atm$ to $torr$

\newpage
\newgeometry{margin=1.25in}
\pagestyle{empty}
\addtocounter{page}{-1}
\section*{\emph{On Shakespeare. 1630}}
\paragraph{By John Milton}~
\begin{verse}
	What needs my Shakespeare for his honoured bones,\\
	The labor of an age in pilèd stones,\\
	Or that his hallowed relics should be hid\\
	Under a star-ypointing pyramid?\\
	Dear son of Memory, great heir of fame,\\
	What need’st thou such weak witness of thy name?\\
	Thou in our wonder and astonishment\\
	Hast built thyself a live-long monument.\\
	For whilst to th’ shame of slow-endeavouring art,\\
	Thy easy numbers flow, and that each heart\\
	Hath from the leaves of thy unvalued book\\
	Those Delphic lines with deep impression took,\\ 
	Then thou, our fancy of itself bereaving,\\
	Dost make us marble with too much conceiving;\\
	And so sepúlchred in such pomp dost lie,\\
	That kings for such a tomb would wish to die.
\end{verse}

\vspace{8em}
\hspace{0.3\linewidth}
\begin{minipage}{0.7\linewidth}
\section*{\emph{Odes III: XXX (23 BCE)}}
\paragraph{Horace}~
\begin{verse}
exegi monumentum aere perennius\\
regalique situ pyramidum altius,\\
quod non imber edax, non Aquilo inpotens\\
possit diruere \ldots\\
Non omnis moriar.

I have built a monument more lasting than bronze,\\
higher than the Pyramids’ regal structures,\\
that no consuming rain, nor wild north wind\\
can destroy \ldots\\
I shall not wholly die.
\end{verse}
\end{minipage}
\end{document}
