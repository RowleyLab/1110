\documentclass[11pt, letterpaper]{memoir}
\usepackage{HomeworkStyle}
\geometry{margin=0.75in}



\begin{document}

	\begin{center}
		{\large Quiz 7.3 --	Equilibrium Reactions}
	\end{center}
	{\large Name: \rule[-1mm]{4in}{.1pt} 

\subsection*{Question 1}
Consider the reaction: \ch{C(s) + H2O(g) <=> CO(g) + H2(g)} \hspace{1em} $\Delta H_{rxn} = 131.3~\dfrac{kJ}{mol}$ \hspace{1em} $K_C=5.63\times10^{-4}$

\noindent
\begin{itemize}
	\item Give the equilibrium expression for this reaction
	
	\vspace{2em}
	\item Is this reaction reactant-favored, or product-favored?
	
	\vspace{2em}
	\item What is $\left[\ch{CO}\right]$ if $\left[\ch{H2O}\right] = 0.100~M$ and $\left[\ch{H2}\right]=2.50\times10^{-3}$
	
	\vspace{2em}
	\item Which way will the reaction shift to restore equilibrium after each of the following changes:
	\begin{itemize}
		\item Remove \ch{H2O(g)}
		\item Add excess \ch{C(s)}
		\item Increase the pressure (reduce system volume)
		\item Increase the temperature
	\end{itemize}
\end{itemize}

\subsection*{Question 2}
Consider the reaction: \ch{H2(g) + Br2(g) <=> 2 HBr(g)}  \hspace{1em} $\Delta H_{rxn} = -72.6~\dfrac{kJ}{mol}$ \hspace{1em} $K_C=62.5$

\noindent
\begin{itemize}
	\item Give the equilibrium expression for this reaction
	
	\vspace{2em}
	\item Is this reaction reactant-favored, or product-favored?
	
	\vspace{2em}
	\item What is $\left[\ch{HBr}\right]$ if $\left[\ch{H2}\right] = 0.0200~M$ and $\left[\ch{Br2}\right]=5.00\times10^{-3}$
	
	\vspace{2em}
	\item Which way will the reaction shift to restore equilibrium after each of the following changes:
	\begin{itemize}
		\item Add \ch{HBr(g)}
		\item Add a catalyst
		\item Increase the pressure (reduce system volume)
		\item Increase the temperature
	\end{itemize}
\end{itemize}
\newpage
\newgeometry{margin=1.25in}
\pagestyle{empty}
\addtocounter{page}{-1}
\section*{\emph{Saint Crispin's Day Speech (from \emph{Henry V}, spoken by King Henry)}}
\paragraph{By William Shakespeare}~
\begin{verse}
	This day is called the feast of Crispian:\\
	He that outlives this day, and comes safe home,\\
	Will stand a tip-toe when the day is named,\\
	And rouse him at the name of Crispian.\\
	He that shall live this day, and see old age,\\
	Will yearly on the vigil feast his neighbours,\\
	And say ‘To-morrow is Saint Crispian:’\\
	Then will he strip his sleeve and show his scars.\\
	And say ‘These wounds I had on Crispin’s day.’\\
	Old men forget: yet all shall be forgot,\\
	But he’ll remember with advantages\\
	What feats he did that day: then shall our names.\\
	Familiar in his mouth as household words\\
	Harry the king, Bedford and Exeter,\\
	Warwick and Talbot, Salisbury and Gloucester,\\
	Be in their flowing cups freshly remember’d.\\
	This story shall the good man teach his son;\\
	And Crispin Crispian shall ne’er go by,\\
	From this day to the ending of the world,\\
	But we in it shall be remember’d;\\
	We few, we happy few, we band of brothers;\\
	For he to-day that sheds his blood with me\\
	Shall be my brother; be he ne’er so vile,\\
	This day shall gentle his condition:\\
	And gentlemen in England now a-bed\\
	Shall think themselves accursed they were not here,\\
	And hold their manhoods cheap whiles any speaks\\
	That fought with us upon Saint Crispin’s day.
\end{verse}
\end{document}
