\documentclass[12pt, letterpaper]{article}
\usepackage{SyllabusStyle}

\begin{document}
\begin{center}
	{\Large \textsc{Elementary Chemistry}}

	CHEM 1110
\end{center}
\begin{center}
	{\large Spring 2022}
\end{center}
\begin{center}
	\rule{0.99\textwidth}{0.4pt}
	\begin{tabular}{llcll}
		\textbf{Instructor:} & Matthew Rowley & & \textbf{Office Hours:} & Daily 10:00 am -- 11:00 am \\
		\textbf{Telephone:} & (435) 586-7875 & & & \\
		\textbf{Email:} & matthewrowley$1$@suu.edu  & & \textbf{Office:} & SC-220\\
		\multicolumn{5}{c}{Please include the course number in the subject line of all correspondence.} 
	\end{tabular}
	\rule{0.99\textwidth}{0.4pt}
\end{center}

\section*{Course Description} 
This course is a general introduction to inorganic chemistry. It is designed for students studying family and consumer sciences, agriculture, nursing, and other students who need only one year of basic chemistry. The course is 3 credits.

\paragraph{Prerequisites:}
A minimum grade of ``C'' (2.0 or above) in MATH 1010 or above.

\paragraph{Concurrent requisite:}
CHEM 1115 -- Elementary Chemistry Lab

\paragraph{Required Course Materials:} ~

\begin{itemize}
\item \emph{Fundamentals of General, Organic, and Bilogical Chemistry} (7th Ed.) by McMurry, Ballantine, Hoeger, and Peterson (ISBN: 978-0-321-75083-9)
\item Access to \emph{Achieve} online homework (purchase through link on Canvas) 
\item A scientific calculator capable of computing logarithms is required.
\end{itemize}

\section*{Tentative Schedule}
This class will meet on Mondays, Wednesdays, and Fridays from 2:00-2:50 in room 114 of the science center (SC):

\noindent For the best lecture experience, read the indicated textbook chapter \emph{before} each lecture.

\noindent
\begin{tabular}{rcccc}
	& Date && Topic & Chapter\\
	\midrule
	Week 1 & M, Jan. 10&& Chemistry: The Central Science & 1.1--1.2\\
	& W, Jan. 12&& Elements and the Periodic Table & 1.3--1.6\\
	& F, Jan. 14&& Measuring Physical Quantitites & 1.7--1.9\\
\end{tabular}

\begin{tabular}{rcccc}
	& Date && Topic & Chapter\\
	\midrule
	Week 2 & M, Jan. 17& \multicolumn{3}{l}{\textbf{Martin Luther King Day -- No Class!}}\\
	& W, Jan. 19&& Numbers and Math in Chemistry & 1.10--1.12\\
	& F, Jan. 21&& Temperature, Heat, and Derived Units & 1.13--1.14\\
	\midrule
	Week 3 & M, Jan. 24&& Atoms, Elements, and Isotopes & 2.1--2.3\\
	& W, Jan. 26&& Atomic Weight, Periodic Table, and Atomic Structure & 2.4--2.6\\
	& F, Jan. 28&& Electron Configuration & 2.7--2.9\\
	\midrule
	Week 4 & M, Jan. 31& \multicolumn{3}{l}{\textbf{Catch-up/Review Day - Midterm Exam 1 (Ch. 1--2)}}\\
	& W, Feb. 2&& Ions and Ionic Bonds & 3.1--3.4\\
	& F, Feb. 4&& Ionic Compounds & 3.5--3.7\\
	\midrule
	Week 5 & M, Feb. 7&& Naming Ionic Compounds & 3.8--3.11\\
	& W, Feb. 9&& Molecular Compounds & 4.1--4.3\\
	& F, Feb. 11&& Covalent Bonds and Molecules & 4.4--4.7\\
	\midrule
	Week 6 & M, Feb. 14&& Molecular Structure & 4.8--4.9\\
	& W, Feb. 16&& Polarity and Binary Molecular Compounds & 4.10--4.11\\
	& F, Feb. 18& \multicolumn{3}{l}{\textbf{Catch-up/Review Day - Midterm Exam 2 (Ch. 3--4)}}\\
	\midrule
	Week 7 & M, Feb. 21& \multicolumn{3}{l}{\textbf{President's Day -- No Class!}}\\
	& W, Feb. 23&& Balancing Chemical Reactions & 5.1--5.3\\
	& F, Feb. 25&& Classes of Chemical Reactions & 5.4--5.6\\
	\midrule
	Week 8 & M, Feb. 28& \multicolumn{3}{l}{\textbf{Spring Break -- No Class!}}\\
	& W, Mar. 2& \multicolumn{3}{l}{\textbf{Spring Break -- No Class!}}\\
	& F, Mar. 4& \multicolumn{3}{l}{\textbf{Spring Break -- No Class!}}\\
	\midrule
	Week 9 & M, Mar. 7&& Redox Reactions & 5.7--5.8\\
	& W, Mar. 9&& Chemical Calculations I & 6.1--6.3\\
	& F, Mar. 11&& Chemical Calculations II & 6.4--6.5\\
\end{tabular}

\begin{tabular}{rcccc}
	& Date && Topic & Chapter\\
	\midrule
	Week 10 & M, Mar. 14&& Chemical Reactions: Energy and Rates & 7.1--7.3\\
	& W, Mar. 16&& Chemical Reactions: Equilibrium & 7.4--7.6\\
	& F, Mar. 18&& Equilibrium Equations & 7.7--7.9\\
	\midrule
	Week 11 & M, Mar. 21& \multicolumn{3}{l}{\textbf{Catch-up/Review Day - Midterm Exam 3 (Ch. 5--7)}}\\
	& W, Mar. 23&& Gases and Kinetic Molecular Theory & 8.1--8.3\\
	& F, Mar. 25&& Pressure and Gas Laws & 8.4--8.7\\
	\midrule
	Week 12 & M, Mar. 28&& Gas Laws & 8.8--8.11\\
	& W, Mar. 30& \multicolumn{3}{l}{\textbf{Festival of Excellence -- No Class!}}\\
	& F, Apr. 1&& Liquids and Solids & 8.12--8.15\\
	\midrule
	Week 13 & M, Apr. 4&& Solutions & 9.1--9.4\\
	& W, Apr. 6&& Solubility and Dilution & 9.5--9.9\\
	& F, Apr. 8&& Ions in Solution: Electrolytes & 9.10--9.13\\
	\midrule
	Week 14 & M, Apr. 11&& Acids and Bases & 10.1--10.5\\
	& W, Apr. 13&& Acids and Bases -- Calculations & 10.6--10.10\\
	& F, Apr. 15&& Buffers and Titrations & 10.11--10.14\\
	\midrule
	Week 15 & M, Apr. 18& \multicolumn{3}{l}{\textbf{Catch-up/Review Day - Midterm Exam 4 (Ch. 8--10)}}\\
	& W, Apr. 20&& Nuclear Chemistry & 11.1--11.5\\
	& F, Apr. 22&& Nuclear Chemistry and Radiation & 11.6--11.11\\
	\midrule
	Finals Week& R, Apr. 28& \multicolumn{3}{l}{\textbf{Final Exam} -- 1:00--2:50 ~~ \emph{Bring a pencil and scantron!}}\\
\end{tabular}

~

\section*{Course Requirements}
Grades will be based on the following items:
\begin{description}
  \item[4 Midterm Exams] 40\%
  \item[Final Exam] 15\%
  \item[Quizzes] 15\%
  \item[Textbook Homework] 15\%
  \item[Achieve Homework] 15\%
\end{description}
Final Grades will be assigned according to the following grade scale:

\begin{tabular}{cl|c|cl}
	Percentage & Grade &  & Percentage & Grade \\ \midrule
	100--93.0 & A     &  &  77.0--73.0 & C     \\
	93.0--90.0 & A-    &  &  73.0--70.0 & C-    \\
	90.0--87.0 & B+    &  &  70.0--67.0 & D+    \\
	87.0--83.0 & B     &  &  67.0--63.0 & D     \\
	83.0--80.0 & B-    &  &  63.0--60.0 & D-    \\
	80.0--77.0 & C+    &  &     < 60.0 & F
\end{tabular}
\paragraph{Midterm Exams:}
There are four midterm exams, to be completed in the testing center during the assigned window unless prior arrangements have been made. It is departmental policy that exams not be returned, although students may examine their finished exam and the answer key in my office.

\paragraph{Final Exam:}
The final exam is a comprehensive and nationally normalized exam prepared by the American Chemical Society.

\paragraph{Quizzes:}
Quizzes will be given at the beginning of most days. The purpose of these quizzes is to provide practice for the exams and to encourage punctual attendance.

\paragraph{Textbook Homework:}
Most days will end with an assignment of a few problems from our textbook. These problems mostly have solutions in the back of the book, so you are encouraged to check your work and try to correct your answers if wrong. The assignments are graded on participation only, and are intended to encourage daily engagement with the material.

\paragraph{Achieve Online Homework:}
The Achieve online homework assignments are organized by chapter and are of substantial length. I recommend completing the assignments in multiple sessions (Achieve saves your work), as we cover new material each day. You can find a link to sign up for Achieve in our Canvas course.

\paragraph{Attendance Policy:}
Students are expected to attend class. If you must miss class, contact the instructor.

\paragraph{Late Work Policy:}
Achieve online homework will be due on Sunday evenings (11:59pm). Students will lose access to the homework assignments once the due date has passed, and they will be graded on the work which was completed by that time. There will be no opportunity to turn in late work.

\paragraph{Make-up Work Policy:}
In general, there will be no opportunity to make up missed work. If you must miss class, please do any assigned work in advance, and arrange to turn it in early.

\section*{Miscellany}

\paragraph{Scientific Calculator:}
There are many different ways to calculate figures during homework. It is tempting to rely on Online resources such as \href{http://www.wolframalpha.com}{http://www.wolframalpha.com}, or to simply use a calculator application on a smart phone. During exams, however, any devices capable of connecting to the Internet will \emph{not} be allowed. You will instead need a scientific calculator capable of performing exponentiation and logarithms for the exams. Using this calculator exclusively while doing homework will ensure that you are familiar with it for use during exams.

\paragraph{Academic Integrity:}
Scholastic dishonesty will not be tolerated and will be prosecuted to the fullest extent. You are expected to have read and understood the current issue of the \href{https://help.suu.edu/handbook}{Student Handbook} (published by Student Services) regarding student responsibilities and rights, and for the intellectual property policy, information about procedures, and what constitutes acceptable behavior. From University policy 6.33: ``The University defines plagiarism as intentionally or carelessly presenting the work of another as one’s own. It includes submitting an assignment purporting to be the student’s original work which has wholly or in part been created by another person, or cutting and pasting of source material\ldots''

\paragraph{ADA Policy:}
Students with medical, psychological, learning, or other disabilities desiring academic adjustments, accommodations, or auxiliary aids will need to contact the Southern Utah University Coordinator of Services for Students with Disabilities (SSD), in Room 206F of the Sharwan Smith Center or phone (435) 865-8022. SSD determines eligibility for and authorizes the provision of services.

\paragraph{Emergency Management Statement:}
In case of emergency, the university's Emergency Notification System (ENS) will be activated. Students are encouraged to maintain updated contact information using the link on the homepage of the \emph{mySUU} portal. In addition, students are encouraged to familiarize themselves with the Emergency Response Protocols posted in each classroom. Detailed information about the university's emergency management plan can be found at: \href{http://www.suu.edu/emergency}{http://www.suu.edu/emergency}

\paragraph{HEOA Compliance Statement:}
The sharing of copyrighted material through peer-to- peer (P2P) file sharing, except as provided under U.S. copyright law, is prohibited by law. Detailed information can be found at: \href{https://help.suu.edu/article/1097/p2p-and-copyright-infringement}{https://help.suu.edu/article/1097/p2p-and-copyright-infringement}

\paragraph{LINK Statement:}
SUU faculty and staff care about the success of our students. In addition to your professor, numerous services are available to assist you with the achievement of your educational goals. SUU's LINK system may be used by faculty to notify you and/or your advisors of their concern for your progress and provide references to campus services as appropriate.

\paragraph{SUUSA Statement:}
As a student at SUU, you have representation from the SUU Student Association (SUUSA) which advocates for student interests and helps work as a liaison between the students and the university administration. You can submit My SUU Voice feedback by going here: \href{https://www.suu.edu/suusa/voice}{https://www.suu.edu/suusa/voice} Likewise, you can learn more about SUUSA's Executive Council here (\href{https://www.suu.edu/suusa/executive-council/}{https://www.suu.edu/suusa/executive-council/}) and about indivdual SUUSA's Student Sentors here (\href{https://www.suu.edu/suusa/senate/}{https://www.suu.edu/suusa/senate/})

\paragraph{University Policies and Recommendations Regarding COVID-19:}
Southern Utah University has compiled a collection of information, policies, and recommendations related to COVID-19 at \href{https://www.suu.edu/coronavirus/}{suu.edu/coronavirus/}

\noindent I dearly want this semester to go smoothly vis-\`a-vis COVID-19, and I assume you all do as well. Toward that end, I encourage you all to exercise all reasonable precaution to prevent the spread of the coronavirus. This includes using the testing and self-reporting resources at the link above.

\noindent It may interest some of you to know that, for my part, I have been vaccinated with two doses of the Moderna vaccine, and more recently received a Pfizer booster. I will wear a mask on campus when appropriate. I will \emph{not} be wearing a mask as I lecture, since clear communication is my primary goal in the classroom.

\paragraph{Disclaimer:}
Information contained in this syllabus, other than the grading, late assignments, make up work and attendance policies, may be subject to change as deemed appropriate by the instructor.
\end{document}