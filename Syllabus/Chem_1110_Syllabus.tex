\documentclass[12pt, letterpaper]{article}
\usepackage{SyllabusStyle}

\begin{document}
\begin{center}
{\Large \textsc{Elementary Chemistry}}

CHEM 1110
\end{center}
\begin{center}
	{\large Spring 2021}
\end{center}
\begin{center}
	\rule{0.85\textwidth}{0.4pt}
	\begin{tabular}{llcll}
		\textbf{Instructor:} & Matthew Rowley & & \textbf{Office Hours:} & Daily 1:00 pm - 2:00 pm \\
		\textbf{Telephone:} & (435) 586-7875 & & \textbf{Office:} & SC-220 \\
		\textbf{Email:} & \multicolumn{3}{l}{matthewrowley$1$@suu.edu}\\
		\multicolumn{5}{c}{Please include the course number in the subject line of all correspondence.} 
	\end{tabular}
	\rule{0.85\textwidth}{0.4pt}
\end{center}

\section*{Course Description} 
This course is a general introduction to inorganic chemistry. It is designed for students studying family and consumer sciences, agriculture, nursing, and other students who need only one year of basic chemistry. The course is 3 credits.

\paragraph{Prerequisites:}
A minimum grade of ``C'' (2.0 or above) in MATH 1010 or above.

\paragraph{Concurrent requisite:}
CHEM 1115 -- Elementary Chemistry Lab

\paragraph{Required Course Materials:} ~

\begin{itemize}
\item \emph{Fundamentals of General, Organic, and Bilogical Chemistry} (7th Ed.) by McMurry, Ballantine, Hoeger, and Peterson (ISBN: 978-0-321-75083-9)
\item Access to \emph{Sapling} online homework (purchase at \href{http://saplinglearning.com}{saplinglearning.com} -- details below) 
\item A scientific calculator capable of computing logarithms is required.
\end{itemize}

\section*{Tentative Schedule}
This class will meet on Mondays, Wednesdays, and Fridays from 2:00 pm to 2:50 pm in room 114 of the Science Center (SC).

\noindent For the best lecture experience, read the indicated textbook chapter \emph{before} each lecture.

\noindent
\begin{tabular}{rcccc}
	& Date && Topic & Chapter\\
	\midrule
	Week 1 & M, Jan. 11&& Chemistry: The Central Science & 1.1--1.2\\
	& W, Jan. 13&& Elements and the Periodic Table & 1.3--1.6\\
	& F, Jan. 15&& Measuring Physical Quantitites & 1.7--1.9\\
\end{tabular}
	
\noindent
\begin{tabular}{rcccc}
	& Date && Topic & Chapter\\
	\midrule
	Week 2 & M, Jan. 18& \multicolumn{3}{l}{\textbf{Martin Luther King Day -- No Class!}}\\
	& W, Jan. 20&& Numbers and Math in Chemistry & 1.10--1.12\\
	& F, Jan. 22&& Temperature, Heat, and Derived Units & 1.13--1.14\\
	\midrule
	Week 3 & M, Jan. 25&& Atoms, Elements, and Isotopes & 2.1--2.3\\
	& W, Jan. 27&& Atomic Weight, Periodic Table, and Atomic Structure & 2.4--2.6\\
	& F, Jan. 29&& Electron Configuration & 2.7--2.9\\
	\midrule
	Week 4 & M, Feb. 1& \multicolumn{3}{l}{\textbf{Catch-up/Review Day}}\\
	& W, Feb. 3&& Ions and Ionic Bonds & 3.1--3.4\\
	& F, Feb. 5&& Ionic Compounds & 3.5--3.7\\
	\midrule
	Week 5 & M, Feb. 8&& Naming Ionic Compounds & 3.8--3.11\\
	& W, Feb. 10&& Molecular Compounds & 4.1--4.3\\
	& F, Feb. 12&& Covalent Bonds and Molecules & 4.4--4.7\\
	\midrule
	Week 6 & M, Feb. 15& \multicolumn{3}{l}{\textbf{President's Day -- No Class!}}\\
	& W, Feb. 17&& Molecular Structure & 4.8--4.9\\
	& F, Feb. 19&& Polarity and Binary Molecular Compounds & 4.10--4.11\\
	\midrule
	Week 7 & M, Feb. 22& \multicolumn{3}{l}{\textbf{Catch-up/Review Day}}\\
	& W, Feb. 24&& Balancing Chemical Reactions & 5.1--5.3\\
	& F, Feb. 26&& Classes of Chemical Reactions & 5.4--5.6\\
	\midrule
	Week 8 & M, Mar. 1& \multicolumn{3}{l}{\textbf{Spring Break -- No Class!}}\\
	& W, Mar. 3& \multicolumn{3}{l}{\textbf{Spring Break -- No Class!}}\\
	& F, Mar. 5& \multicolumn{3}{l}{\textbf{Spring Break -- No Class!}}\\
	\midrule
	Week 9 & M, Mar. 8&& Redox Reactions & 5.7--5.8\\
	& W, Mar. 10&& Chemical Calculations I & 6.1--6.3\\
	& F, Mar. 12&& Chemical Calculations II & 6.4--6.5\\
\end{tabular}

\noindent
\begin{tabular}{rcccc}
& Date && Topic & Chapter\\
	\midrule
	Week 10 & M, Mar. 15& \multicolumn{3}{l}{\textbf{Catch-up/Review Day}}\\
	& W, Mar. 17&& Chemical Reactions: Energy and Rates & 7.1--7.3\\
	& F, Mar. 19&& Chemical Reactions: Equilibrium & 7.4--7.6\\
	\midrule
	Week 11 & M, Mar. 22&& Equilibrium Equations & 7.7--7.9\\
	& W, Mar. 24&& Gases and Kinetic Molecular Theory & 8.1--8.3\\
	& F, Mar. 26&& Pressure and Gas Laws & 8.4--8.7\\
	\midrule
	Week 12 & M, Mar. 29&& Gas Laws & 8.8--8.11\\
	& W, Mar. 31& \multicolumn{3}{l}{\textbf{Festival of Excellence -- No Class!}}\\
	& F, Apr. 2&& Liquids and Solids & 8.12--8.15\\
	\midrule
	Week 13 & M, Apr. 5&& Solutions & 9.1--9.4\\
	& W, Apr. 7&& Solubility and Dilution & 9.5--9.9\\
	& F, Apr. 9&& Ions in Solution: Electrolytes & 9.10--9.13\\
	\midrule
	Week 14 & M, Apr. 12& \multicolumn{3}{l}{\textbf{Catch-up/Review Day}}\\
	& W, Apr. 14&& Acids and Bases & 10.1--10.5\\
	& F, Apr. 16&& Acids and Bases -- Calculations & 10.6--10.10\\
	\midrule
	Week 15 & M, Apr. 19&& Buffers and Titrations & 10.11--10.14\\
	& W, Apr. 21&& Nuclear Chemistry & 11.1--11.5\\
	& F, Apr. 23&& Nuclear Chemistry and Radiation & 11.6--11.11\\
	\midrule
	Finals Week& R, May 2& \multicolumn{3}{l}{\textbf{Final Exam --- Details will be given closer to finals week}}\\
\end{tabular}

~

\section*{Course Requirements}
Grades will be based on the following items:
\begin{description}
  \item[4 Midterm Exams] 40\%
  \item[Final Exam] 15\%
  \item[Quizzes] 15\%
  \item[Homework] 30\%
\end{description}
Final Grades will be assigned according to the following grade scale:

\begin{tabular}{cl|c|cl}
	Percentage & Grade &  & Percentage & Grade \\ \midrule
	100--93.0 & A     &  &  77.0--73.0 & C     \\
	93.0--90.0 & A-    &  &  73.0--70.0 & C-    \\
	90.0--87.0 & B+    &  &  70.0--67.0 & D+    \\
	87.0--83.0 & B     &  &  67.0--63.0 & D     \\
	83.0--80.0 & B-    &  &  63.0--60.0 & D-    \\
	80.0--77.0 & C+    &  &     < 60.0 & F
\end{tabular}
\paragraph{Midterm Exams:}
There are four midterm exams, to be completed in the testing center during the assigned window unless prior arrangements have been made. It is departmental policy that exams not be returned, although students may examine their finished exam and the answer key in my office.

\paragraph{Final Exam:}
The final exam is comprehensive. It may be administered over Canvas to accommodate remote instruction this year.

\paragraph{Sapling Online Homework:}
The Sapling Learning online homework assignments are organized by chapter and are of substantial length. I recommend completing the assignments in multiple sessions (Sapling saves your work), as we cover new material each day. Several of the assignments are listed as trainings, and will help to familiarize you with the Sapling user interface and proper formatting of answers.  General instructions from Sapling can be found at: \newline\href{ http://bit.ly/saplinginstructions}{ http://bit.ly/saplinginstructions}. 

\noindent To register for access to the Sapling homework problems for this course:
\begin{enumerate}
  \item Go to \href{http://www2.saplinglearning.com/}{http://www2.saplinglearning.com/} and click on ``US Higher Ed'' at the top right.
  \item \begin{enumerate}
  \item If you already have a Sapling Learning account, log in and skip to step 3.
  \item If you have a Facebook account, you can use it to quickly create a Sapling Learning account. Click on ``Create an Account,'' then ``Create my account through Facebook.'' You will be prompted to log into Facebook if you aren't already. Choose a username and password, then click ``Link Account.'' You can then proceed to step 3.
  \item Otherwise, click ``Create an Account,'' supply the requested information, and then click ``Create My Account.'' Check your email for a message from Sapling Learning and follow the link provided in that email to activate your account.
  \end{enumerate}
  \item Find your course in the list (you may need to expand the subject and term categories) and click the link (Southern Utah University - CHEM 1110 - Spring21 - ROWLEY)
  \item Select a payment option and follow the remaining instructions.
\end{enumerate}

\paragraph{Quizzes:}
Quizzes will be posted on Canvas and are graded automatically. The purpose of these quizzes is to encourage daily engagement with the material as well as to provide practice for the exams.

\paragraph{Attendance Policy:}
Students are expected to attend class. If you must miss class, contact the instructor.

\paragraph{Late Work Policy:}
Sapling Learning online homework will be due on Sunday evenings (11:59pm). Students will lose access to the homework assignments once the due date has passed, and they will be graded on the work which was completed by that time. There will be no opportunity to turn in late work.

\paragraph{Make-up Work Policy:}
In general, there will be no opportunity to make up missed work. If you must miss class, please do any assigned work in advance, and arrange to turn it in early.

\section*{Miscellany}
\paragraph{Important syllabus statements related to ATTENDANCE and COVID-19:}
\begin{description}
	\item[Q:~~~~~~] If a student does NOT want to attend face-to-face this fall, is there another option?
	\item[A:~~~~~~] All students who would normally attend face to face classes should plan to do so, unless they are ill or are concerned for their health. However, if a student is ill or concerned for their health, students will be able to complete classes this fall whether they stay home or are here in Cedar City. Digital recording equipment will be installed in each instructional space (classrooms, labs and other venues) so that students can log in and attend face-to-face classes remotely. This will allow any SUU community member to engage in classroom activities in the way they feel comfortable. Faculty will also be able to teach remotely in accordance with their personal health needs. Faculty will likely post the link for how to listen/watch live in the syllabus (and/or Canvas) or simply email it to their students. We recommend to students that they reach out to their faculty to notify them how they will be 'attending' class.
	\item[Q:~~~~~~] Would it be possible to participate in courses remotely for the entire fall 2020 semester if desired?
	\item[A:~~~~~~] Yes. The intent is that this semester is normal and that students attend face to face classes, however, if they are symptomatic or concerned for their health, they can participate remotely. All students will be able to complete the fall semester from another location. Students should notify their faculty if they are completing the course from a distance rather than attending class in-person. International students will need to be mindful of how this will impact their non-immigrant status and to contact International Affairs if there are any questions.
	\item[Q:~~~~~~] Do students need special documentation to be allowed to attend classes remotely?
	\item[A:~~~~~~]  No, students do not need special documentation to attend classes remotely. However, students should notify their faculty if they are completing the course from a distance rather than attending class in-person so expectations can be clearly understood. Remote attendance should be for those who are ill or symptomatic.
\end{description}

\paragraph{ZOOM ETIQUETTE:}
Your class may utilize the Zoom online conference system. To participate in Zoom meetings, you will need to have a webcam/microphone or a smartphone with the Zoom app. We will adopt the same rules and norms as in a physical classroom (take notes; participate by asking and answering questions; wear classroom-ready clothing). For everyone’s benefit:
\begin{itemize}
	\item Join the course in a quiet, distraction free location
	\item Be aware of your background
	\item Turn on your video (you may close it after attendance is taken if your internet connection cannot handle having both audio and video going)
	\item Mute your microphone unless you are speaking
	\item Close browser tabs and software not required for participating in class
	\item Remember that our classes are in the Mountain Time zone
\end{itemize}
The success of this class will depend on the same commitment to learning we all typically bring to the physical classroom.

\paragraph{Scientific Calculator:}
There are many different ways to calculate figures during homework. It is tempting to rely on Online resources such as \href{http://www.wolframalpha.com}{http://www.wolframalpha.com}, or to simply use a calculator application on a smart phone. During exams, however, any devices capable of connecting to the Internet will \emph{not} be allowed. You will instead need a scientific calculator capable of performing exponentiation and logarithms for the exams. Using this calculator exclusively while doing homework will ensure that you are familiar with it for use during exams.

\paragraph{Academic Integrity:}
Scholastic dishonesty will not be tolerated and will be prosecuted to the fullest extent. You are expected to have read and understood the current issue of the \href{https://help.suu.edu/handbook}{Student Handbook} (published by Student Services) regarding student responsibilities and rights, and for the intellectual property policy, information about procedures, and what constitutes acceptable behavior. From University policy 6.33: ``The University defines plagiarism as intentionally or carelessly presenting the work of another as one’s own. It includes submitting an assignment purporting to be the student’s original work which has wholly or in part been created by another person, or cutting and pasting of source material\ldots''

\paragraph{ADA Policy:}
Students with medical, psychological, learning, or other disabilities desiring academic adjustments, accommodations, or auxiliary aids will need to contact the Southern Utah University Coordinator of Services for Students with Disabilities (SSD), in Room 206F of the Sharwan Smith Center or phone (435) 865-8022. SSD determines eligibility for and authorizes the provision of services.

\paragraph{Emergency Management Statement:}
In case of emergency, the university's Emergency Notification System (ENS) will be activated. Students are encouraged to maintain updated contact information using the link on the homepage of the \emph{mySUU} portal. In addition, students are encouraged to familiarize themselves with the Emergency Response Protocols posted in each classroom. Detailed information about the university's emergency management plan can be found at: \href{http://www.suu.edu/emergency}{http://www.suu.edu/emergency}

\paragraph{HEOA Compliance Statement:}
The sharing of copyrighted material through peer-to- peer (P2P) file sharing, except as provided under U.S. copyright law, is prohibited by law. Detailed information can be found at: \href{https://help.suu.edu/article/1097/p2p-and-copyright-infringement}{https://help.suu.edu/article/1097/p2p-and-copyright-infringement}

\paragraph{LINK Statement:}
SUU faculty and staff care about the success of our students. In addition to your professor, numerous services are available to assist you with the achievement of your educational goals. SUU's LINK system may be used by faculty to notify you and/or your advisors of their concern for your progress and provide references to campus services as appropriate.

\paragraph{SUUSA Statement:}
As a student at SUU, you have representation from the SUU Student Association (SUUSA) which advocates for student interests and helps work as a liaison between the students and the university administration. You can submit My SUU Voice feedback by going here: \href{https://www.suu.edu/suusa/voice}{https://www.suu.edu/suusa/voice} Likewise, you can learn more about SUUSA's Executive Council here (\href{https://www.suu.edu/suusa/executive-council/}{https://www.suu.edu/suusa/executive-council/}) and about indivdual SUUSA's Student Sentors here (\href{https://www.suu.edu/suusa/senate/}{https://www.suu.edu/suusa/senate/})

\paragraph{Disclaimer:}
Information contained in this syllabus, other than the grading, late assignments, make up work and attendance policies, may be subject to change as deemed appropriate by the instructor.
\end{document}