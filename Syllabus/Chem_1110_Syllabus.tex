\documentclass[12pt, letterpaper]{article}
\usepackage{SyllabusStyle}

\begin{document}
\begin{center}
	{\Large \textsc{Elementary Chemistry}}

	CHEM 1110
\end{center}
\begin{center}
	{\large Fall 2023}
\end{center}
\begin{center}
	\rule{0.99\textwidth}{0.4pt}
	\begin{tabular}{llcll}
		\textbf{Instructor:} & Matthew Rowley           &  & \textbf{Office Hours:} & MWRF 10:00 am -- 11:00 am \\
		\textbf{Telephone:}  & (435) 586-7875           &  &                        & T 1:00 pm -- 2:00 pm    \\
		\textbf{Email:}      & matthewrowley$1$@suu.edu &  & \textbf{Office:}       & SC-220                   \\
		\multicolumn{5}{c}{Please include the course number in the subject line of all correspondence.}
	\end{tabular}
	\rule{0.99\textwidth}{0.4pt}
\end{center}

\section*{Course Description}
This course is a general introduction to inorganic chemistry. It is designed for students studying family and consumer sciences, agriculture, nursing, and other students who need only one year of basic chemistry. The course is 3 credits.

\paragraph{Prerequisites:}
A minimum grade of ``C'' (2.0 or above) in MATH 1010 or above.

\paragraph{Concurrent requisite:}
CHEM 1115 -- Elementary Chemistry Lab

\paragraph{Required Course Materials:} ~

\begin{itemize}
	\item \emph{Fundamentals of General, Organic, and Bilogical Chemistry} (8th Ed.) by McMurry, Ballantine, Hoeger, and Peterson (ISBN: 978-0-13-401518-7)
	\item Access to \emph{Achieve} online homework (purchase through link on Canvas)
	\item A scientific calculator capable of computing logarithms is required.
\end{itemize}

\section*{Tentative Schedule}
This class will meet on Mondays, Wednesdays, and Fridays from 11:00  -- 11:50
am in room 214 of the science center (SC).


\noindent For the best lecture experience, read the indicated textbook chapter \emph{before} each lecture.

\noindent
\begin{tabular}{rcccc}
& Date && Topic & Chapter\\
\midrule
Week 1 & W, Aug. 30&& Chemistry: The Central Science & 1.1--1.2\\
& F, Sep. 1&& Elements and the Periodic Table & 1.3--1.5\\
\end{tabular}

\noindent
\begin{tabular}{rcccc}
& Date && Topic & Chapter\\
\midrule
Week 2 & M, Sep. 4& \multicolumn{3}{l}{\textbf{Labor Day -- No Class!}}\\
& \multirow{2}{*}{W, Sep. 6}& & Measuring Physical Quantitites & 1.6--1.8\\
& & & Numbers and Math in Chemistry & 1.9--1.10\\
& F, Sep. 8&& Temperature, Heat, and Derived Units & 1.11--1.12\\
\midrule
Week 3 & M, Sep. 11&& Atoms, Elements, and Isotopes & 2.1--2.3\\
& W, Sep. 13&& Atomic Weight, Periodic Table, and Atomic Structure & 2.4--2.6\\
& F, Sep. 15&& Electron Configuration & 2.7--2.9\\
\midrule
Week 4 & M, Sep. 18& \multicolumn{3}{l}{\textbf{Catch-up/Review Day - Midterm Exam 1 (Ch. 1--2)}}\\
& W, Sep. 20&& Monoatomic Ions & 3.1--3.4\\
& F, Sep. 22&& Polyatomic Ions & 3.5--3.7\\
\midrule
Week 5 & M, Sep. 25&& Ionic Compounds & 3.8--3.11\\
& W, Sep. 27&& Molecular Compounds & 4.1--4.3\\
& F, Sep. 29&& Covalent Bonds and Molecules & 4.4--4.7\\
\midrule
Week 6 & M, Oct. 2&& Molecular Structure & 4.8--4.9\\
& W, Oct. 4&& Polarity and Binary Molecular Compounds & 4.10--4.11\\
& F, Oct. 6& \multicolumn{3}{l}{\textbf{Catch-up/Review Day - Midterm Exam 2 (Ch. 3--4)}}\\
\midrule
Week 7 & M, Oct. 9&& Balancing Chemical Reactions & 5.1--5.2\\
& W, Oct. 11&& Solubility and Acid/Base Reactions & 5.3--5.4\\
& F, Oct. 13&& Redox Reactions & 5.5--5.7\\
\midrule
Week 8 & M, Oct. 16& \multicolumn{3}{l}{\textbf{Fall Break -- No Class!}}\\
& W, Oct. 18&& Chemical Calculations I & 6.1--6.3\\
& F, Oct. 20&& Chemical Calculations II & 6.4--6.5\\
\midrule
Week 9 & M, Oct. 23&& Chemical Reactions: Energy and Rates & 7.1--7.3\\
& W, Oct. 25&& Chemical Reactions: Equilibrium & 7.4--7.6\\
& F, Oct. 27&& Equilibrium Equations & 7.7--7.9\\
\end{tabular}

\noindent
\begin{tabular}{rcccc}
& Date && Topic & Chapter\\
\midrule
Week 10 & M, Oct. 30& \multicolumn{3}{l}{\textbf{Catch-up/Review Day - Midterm Exam 3 (Ch. 5--7)}}\\
& W, Nov. 1&& Gases and Kinetic Molecular Theory & 8.1--8.3\\
& F, Nov. 3&& Pressure and Gas Laws & 8.4--8.7\\
\midrule
Week 11 & M, Nov. 6&& Gas Laws & 8.8--8.11\\
& W, Nov. 8&& Liquids and Solids & 8.12--8.14\\
& F, Nov. 10&& Solutions & 9.1--9.3\\
\midrule
Week 12 & M, Nov. 13&& Solubility and Dilution & 9.4--9.8\\
& W, Nov. 15&& Electrolyte Solutions & 9.9--9.11\\
& F, Nov. 17&& Acids and Bases & 10.1--10.2\\
\midrule
Week 13 & M, Nov. 20& \multicolumn{3}{l}{\textbf{Thanksgiving Break -- No Class!}}\\
& W, Nov. 22& \multicolumn{3}{l}{\textbf{Thanksgiving Break -- No Class!}}\\
& F, Nov. 24& \multicolumn{3}{l}{\textbf{Thanksgiving Break -- No Class!}}\\
\midrule
Week 14 & M, Nov. 27&& Acids and Bases -- Calculations & 10.3--10.8\\
& W, Nov. 29&& Buffers and Titrations & 10.9--10.11\\
& F, Dec. 1& \multicolumn{3}{l}{\textbf{Catch-up/Review Day - Midterm Exam 4 (Ch. 8--10)}}\\
\midrule
Week 15 & M, Dec. 4&& Nuclear Chemistry & 11.1--11.5\\
& W, Dec. 6&& Nuclear Chemistry and Radiation & 11.6--11.9\\
& F, Dec. 8& \multicolumn{3}{l}{\textbf{Catch-up/Review Day - Final Exam}}\\
\midrule
Finals Week& T, Dec 12& \multicolumn{3}{l}{\textbf{Final Exam 11:00--12:50 pm: Bring a pencil and scantron}}\\
\end{tabular}
~

\section*{Course Requirements}
Grades will be based on the following items:
\begin{description}
	\item[4 Midterm Exams] 40\%
	\item[Final Exam] 20\%
	\item[Quizzes] 10\%
	\item[Textbook Homework] 15\%
	\item[Achieve Homework] 15\%
\end{description}
Final Grades will be assigned according to the following grade scale:

\begin{tabular}{cl|c|cl}
	Percentage & Grade &  & Percentage & Grade \\ \midrule
	100--93.0  & A     &  & 77.0--73.0 & C     \\
	93.0--90.0 & A-    &  & 73.0--70.0 & C-    \\
	90.0--87.0 & B+    &  & 70.0--67.0 & D+    \\
	87.0--83.0 & B     &  & 67.0--63.0 & D     \\
	83.0--80.0 & B-    &  & 63.0--60.0 & D-    \\
	80.0--77.0 & C+    &  & < 60.0     & F
\end{tabular}
\paragraph{Midterm Exams:}
There are four midterm exams, to be completed in the testing center during the assigned window unless prior arrangements have been made. It is departmental policy that exams not be returned, although students may examine their finished exam and the answer key in my office.

\paragraph{Final Exam:}
The final exam is a comprehensive and nationally normalized exam prepared by the American Chemical Society.

\paragraph{Quizzes:}
Quizzes will be given at the beginning of most days. The purpose of these quizzes is to provide practice for the exams and to encourage punctual attendance.

\paragraph{Textbook Homework:}
Most days will end with an assignment of a few problems from our textbook. These problems mostly have solutions in the back of the book, so you are encouraged to check your work and try to correct your answers if wrong. The assignments are graded on participation only, and are intended to encourage daily engagement with the material.

\paragraph{Achieve Online Homework:}
The Achieve online homework assignments are organized by chapter and are of substantial length. I recommend completing the assignments in multiple sessions (Achieve saves your work), as we cover new material each day. You can find a link to sign up for Achieve in our Canvas course.

\paragraph{Attendance Policy:}
Students are expected to attend class. If you must miss class, contact the instructor.

\paragraph{Late Work Policy:}
Achieve online homework will be due on Sunday evenings (11:59pm). Students will lose access to the homework assignments once the due date has passed, and they will be graded on the work which was completed by that time. There will be no opportunity to turn in late work.

\paragraph{Make-up Work Policy:}
In general, there will be no opportunity to make up missed work. If you must miss class, please do any assigned work in advance, and arrange to turn it in early.

\section*{Miscellany}

\paragraph{Important syllabus statements related to ATTENDANCE and COVID-19:} ~

\noindent\emph{What should I expect in the classroom this semester?}

\noindent The following are general guidelines for the classroom environment
\begin{description}
	\item[Class Attendance is Required:] If you are registered for a Face-to-Face, Synchronous Remote, or Hybrid course, attendance is required. If you are ill or instructed to isolate or quarantine, you may request a faculty member record the class and share it with you or you may request other reasonable accommodations. Your instructor will work with you to develop a plan for completing coursework while you are isolated/quarantined. In order for you to receive academic accommodations and ensure that your request is communicated to faculty, you must submit this \href{https://my.suu.edu/covid/selfreport/}{self report form}.
	\item[\href{https://www.suu.edu/registrar/onlinehybrid.html}{Course ~delivery ~modalities} ~are ~posted ~online ~for ~each ~course, ~but ~may ~be ~modified ~in] \textbf{response to emerging COVID conditions:} SUU is employing every effort to maintain a learning environment that is engaging and safe. The course modality listed when you registered for courses should remain for the semester; however, due to COVID conditions, the delivery of modality for a specific course may change during the semester. Normally, these changes will be short term (possibly the length of a quarantine or isolation time period), or in some cases longer. When such a modification is needed, faculty members will work with their department chair and/or dean and the students to maintain an effective learning environment.
\end{description}

\paragraph{Scientific Calculator:}
There are many different ways to calculate figures during homework. It is tempting to rely on Online resources such as \href{http://www.wolframalpha.com}{http://www.wolframalpha.com}, or to simply use a calculator application on a smart phone. During exams, however, any devices capable of connecting to the Internet will \emph{not} be allowed. You will instead need a scientific calculator capable of performing exponentiation and logarithms for the exams. Using this calculator exclusively while doing homework will ensure that you are familiar with it for use during exams.

\paragraph{Academic Integrity:}
Scholastic dishonesty will not be tolerated and will be prosecuted to the fullest extent (see \href{https://www.suu.edu/policies/06/33.html}{SUU Policy 6.33}). You are expected to have read and understood the current SUU student conduct code (\href{https://www.suu.edu/policies/11/02.html}{SUU Policy 11.2}) regarding student responsibilities and rights, the intellectual property policy (\href{https://www.suu.edu/policies/05/52.html}{SUU Policy 5.52}), information about procedures, and what constitutes acceptable behavior.

\paragraph{Mental Health:}
Mental and physical health are equal components to a holistic view of wellness and human thriving. Mental health should not be ignored, dismissed, or demeaned. If you find yourself struggling with mental health please visit \href{https://www.suu.edu/mentalhealth}{https://www.suu.edu/mentalhealth} for resources. There is also a link prominently on the right side of every Canvas page.

\paragraph{ADA Policy:}
Students with medical, psychological, learning, or other disabilities desiring academic adjustments, accommodations, or auxiliary aids will need to contact the Southern Utah University Coordinator of Services for Students with Disabilities (SSD), in Room 206F of the Sharwan Smith Center or phone (435) 865-8022. SSD determines eligibility for and authorizes the provision of services.

\paragraph{Emergency Management Statement:}
In case of emergency, the university's Emergency Notification System (ENS) will be activated. Students are encouraged to maintain updated contact information using the link on the homepage of the \emph{mySUU} portal. In addition, students are encouraged to familiarize themselves with the Emergency Response Protocols posted in each classroom. Detailed information about the university's emergency management plan can be found at: \href{http://www.suu.edu/emergency}{http://www.suu.edu/emergency}

\paragraph{HEOA Compliance Statement:}
The sharing of copyrighted material through peer-to- peer (P2P) file sharing, except as provided under U.S. copyright law, is prohibited by law. Detailed information can be found at: \href{https://help.suu.edu/article/1097/p2p-and-copyright-infringement}{https://help.suu.edu/article/1097/p2p-and-copyright-infringement}

\paragraph{LINK Statement:}
SUU faculty and staff care about the success of our students. In addition to your professor, numerous services are available to assist you with the achievement of your educational goals. SUU's LINK system may be used by faculty to notify you and/or your advisors of their concern for your progress and provide references to campus services as appropriate.

\paragraph{SUUSA Statement:}
As a student at SUU, you have representation from the SUU Student Association (SUUSA) which advocates for student interests and helps work as a liaison between the students and the university administration. You can submit My SUU Voice feedback by going here: \href{https://www.suu.edu/suusa/voice}{https://www.suu.edu/suusa/voice} Likewise, you can learn more about SUUSA's Executive Council here (\href{https://www.suu.edu/suusa/executive-council/}{https://www.suu.edu/suusa/executive-council/}) and about individual SUUSA's Student Senators here (\href{https://www.suu.edu/suusa/senate/}{https://www.suu.edu/suusa/senate/})

\paragraph{Land Acknowledgement Statement:}
SUU wishes to acknowledge and honor the Indigenous communities of this region as original possessors, stewards, and inhabitants of this Too’veep (land), and recognize that the University is situated on the traditional homelands of the Nung’wu (Southern Paiute People). We recognize that these lands have deeply rooted spiritual, cultural, and historical significance to the Southern Paiutes. We offer gratitude for the land itself, for the collaborative and resilient nature of the Southern Paiute people, and for the continuous opportunity to study, learn, work, and build community on their homelands here today. Consistent with the University's ongoing commitment to equity, diversity, and inclusion, SUU works towards building meaningful relationships with Native Nations and Indigenous communities through academic pursuits, partnerships, historical recognitions, community service, and student success efforts.

\paragraph{Disclaimer:}
Information contained in this syllabus, other than the grading, late assignments, make up work and attendance policies, may be subject to change as deemed appropriate by the instructor.
\end{document}
